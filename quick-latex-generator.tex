\documentclass[]{article}
\usepackage{lmodern}
\usepackage{amssymb,amsmath}
\usepackage{ifxetex,ifluatex}
\usepackage{fixltx2e} % provides \textsubscript
\ifnum 0\ifxetex 1\fi\ifluatex 1\fi=0 % if pdftex
  \usepackage[T1]{fontenc}
  \usepackage[utf8]{inputenc}
\else % if luatex or xelatex
  \ifxetex
    \usepackage{mathspec}
  \else
    \usepackage{fontspec}
  \fi
  \defaultfontfeatures{Ligatures=TeX,Scale=MatchLowercase}
\fi
% use upquote if available, for straight quotes in verbatim environments
\IfFileExists{upquote.sty}{\usepackage{upquote}}{}
% use microtype if available
\IfFileExists{microtype.sty}{%
\usepackage{microtype}
\UseMicrotypeSet[protrusion]{basicmath} % disable protrusion for tt fonts
}{}
\usepackage[margin=1in]{geometry}
\usepackage{hyperref}
\hypersetup{unicode=true,
            pdftitle={raw sensor data},
            pdfauthor={Jill Daly},
            pdfborder={0 0 0},
            breaklinks=true}
\urlstyle{same}  % don't use monospace font for urls
\usepackage{color}
\usepackage{fancyvrb}
\newcommand{\VerbBar}{|}
\newcommand{\VERB}{\Verb[commandchars=\\\{\}]}
\DefineVerbatimEnvironment{Highlighting}{Verbatim}{commandchars=\\\{\}}
% Add ',fontsize=\small' for more characters per line
\usepackage{framed}
\definecolor{shadecolor}{RGB}{248,248,248}
\newenvironment{Shaded}{\begin{snugshade}}{\end{snugshade}}
\newcommand{\AlertTok}[1]{\textcolor[rgb]{0.94,0.16,0.16}{#1}}
\newcommand{\AnnotationTok}[1]{\textcolor[rgb]{0.56,0.35,0.01}{\textbf{\textit{#1}}}}
\newcommand{\AttributeTok}[1]{\textcolor[rgb]{0.77,0.63,0.00}{#1}}
\newcommand{\BaseNTok}[1]{\textcolor[rgb]{0.00,0.00,0.81}{#1}}
\newcommand{\BuiltInTok}[1]{#1}
\newcommand{\CharTok}[1]{\textcolor[rgb]{0.31,0.60,0.02}{#1}}
\newcommand{\CommentTok}[1]{\textcolor[rgb]{0.56,0.35,0.01}{\textit{#1}}}
\newcommand{\CommentVarTok}[1]{\textcolor[rgb]{0.56,0.35,0.01}{\textbf{\textit{#1}}}}
\newcommand{\ConstantTok}[1]{\textcolor[rgb]{0.00,0.00,0.00}{#1}}
\newcommand{\ControlFlowTok}[1]{\textcolor[rgb]{0.13,0.29,0.53}{\textbf{#1}}}
\newcommand{\DataTypeTok}[1]{\textcolor[rgb]{0.13,0.29,0.53}{#1}}
\newcommand{\DecValTok}[1]{\textcolor[rgb]{0.00,0.00,0.81}{#1}}
\newcommand{\DocumentationTok}[1]{\textcolor[rgb]{0.56,0.35,0.01}{\textbf{\textit{#1}}}}
\newcommand{\ErrorTok}[1]{\textcolor[rgb]{0.64,0.00,0.00}{\textbf{#1}}}
\newcommand{\ExtensionTok}[1]{#1}
\newcommand{\FloatTok}[1]{\textcolor[rgb]{0.00,0.00,0.81}{#1}}
\newcommand{\FunctionTok}[1]{\textcolor[rgb]{0.00,0.00,0.00}{#1}}
\newcommand{\ImportTok}[1]{#1}
\newcommand{\InformationTok}[1]{\textcolor[rgb]{0.56,0.35,0.01}{\textbf{\textit{#1}}}}
\newcommand{\KeywordTok}[1]{\textcolor[rgb]{0.13,0.29,0.53}{\textbf{#1}}}
\newcommand{\NormalTok}[1]{#1}
\newcommand{\OperatorTok}[1]{\textcolor[rgb]{0.81,0.36,0.00}{\textbf{#1}}}
\newcommand{\OtherTok}[1]{\textcolor[rgb]{0.56,0.35,0.01}{#1}}
\newcommand{\PreprocessorTok}[1]{\textcolor[rgb]{0.56,0.35,0.01}{\textit{#1}}}
\newcommand{\RegionMarkerTok}[1]{#1}
\newcommand{\SpecialCharTok}[1]{\textcolor[rgb]{0.00,0.00,0.00}{#1}}
\newcommand{\SpecialStringTok}[1]{\textcolor[rgb]{0.31,0.60,0.02}{#1}}
\newcommand{\StringTok}[1]{\textcolor[rgb]{0.31,0.60,0.02}{#1}}
\newcommand{\VariableTok}[1]{\textcolor[rgb]{0.00,0.00,0.00}{#1}}
\newcommand{\VerbatimStringTok}[1]{\textcolor[rgb]{0.31,0.60,0.02}{#1}}
\newcommand{\WarningTok}[1]{\textcolor[rgb]{0.56,0.35,0.01}{\textbf{\textit{#1}}}}
\usepackage{graphicx,grffile}
\makeatletter
\def\maxwidth{\ifdim\Gin@nat@width>\linewidth\linewidth\else\Gin@nat@width\fi}
\def\maxheight{\ifdim\Gin@nat@height>\textheight\textheight\else\Gin@nat@height\fi}
\makeatother
% Scale images if necessary, so that they will not overflow the page
% margins by default, and it is still possible to overwrite the defaults
% using explicit options in \includegraphics[width, height, ...]{}
\setkeys{Gin}{width=\maxwidth,height=\maxheight,keepaspectratio}
\IfFileExists{parskip.sty}{%
\usepackage{parskip}
}{% else
\setlength{\parindent}{0pt}
\setlength{\parskip}{6pt plus 2pt minus 1pt}
}
\setlength{\emergencystretch}{3em}  % prevent overfull lines
\providecommand{\tightlist}{%
  \setlength{\itemsep}{0pt}\setlength{\parskip}{0pt}}
\setcounter{secnumdepth}{5}
% Redefines (sub)paragraphs to behave more like sections
\ifx\paragraph\undefined\else
\let\oldparagraph\paragraph
\renewcommand{\paragraph}[1]{\oldparagraph{#1}\mbox{}}
\fi
\ifx\subparagraph\undefined\else
\let\oldsubparagraph\subparagraph
\renewcommand{\subparagraph}[1]{\oldsubparagraph{#1}\mbox{}}
\fi

%%% Use protect on footnotes to avoid problems with footnotes in titles
\let\rmarkdownfootnote\footnote%
\def\footnote{\protect\rmarkdownfootnote}

%%% Change title format to be more compact
\usepackage{titling}

% Create subtitle command for use in maketitle
\newcommand{\subtitle}[1]{
  \posttitle{
    \begin{center}\large#1\end{center}
    }
}

\setlength{\droptitle}{-2em}
  \title{raw sensor data}
  \pretitle{\vspace{\droptitle}\centering\huge}
  \posttitle{\par}
  \author{Jill Daly}
  \preauthor{\centering\large\emph}
  \postauthor{\par}
  \predate{\centering\large\emph}
  \postdate{\par}
  \date{01 December, 2018}


\begin{document}
\maketitle

{
\setcounter{tocdepth}{2}
\tableofcontents
}
\begin{Shaded}
\begin{Highlighting}[]
\CommentTok{#' Take a list of sensor dataframes and re-postion the columns}
\CommentTok{#' to correct the XYZ ordering for each sensor. This accounts }
\CommentTok{#' for the different orientation for each sensor axis.}
\CommentTok{#' }
\CommentTok{#' @param sensor_list a list of dataframe objects}
\CommentTok{#' @param reordered_header a vector of the true XYZ ordering}
\CommentTok{#' for each sensor}
\CommentTok{#' @return a list of dataframes with a true XYZ ordering of }
\CommentTok{#' sensor data}
\CommentTok{#' @example }
\CommentTok{#' convert_axis(senosr_ds_list, c("1_Z", "1_X", "1_Y", "2_Y",...))}
\NormalTok{convert_axis <-}\StringTok{ }\ControlFlowTok{function}\NormalTok{(sensor_list, mag_hdr, mag_negate_hdr, orient_hdr, orient_negate_hdr) \{}
  
\NormalTok{  abs_swap <-}\StringTok{ }\ControlFlowTok{function}\NormalTok{(df_name, mag_swap, mag_negate, orient_swap, orient_negate, sd_list) \{}

    \CommentTok{# Extract the dataframe}
\NormalTok{    df <-}\StringTok{ }\NormalTok{sd_list[[df_name]]}
    
    \CommentTok{# 1. Take a copy of the original header}
\NormalTok{    o_hdr <-}\StringTok{ }\KeywordTok{colnames}\NormalTok{(df)}
    
    \CommentTok{# 2. Apply X, Y col swap & Negate Z for Mag axes only}
    \ControlFlowTok{if}\NormalTok{ (}\KeywordTok{str_detect}\NormalTok{(df_name, }\StringTok{"mag"}\NormalTok{)) \{}
      \CommentTok{# 2a. Swap Raw Magnetometer Col1 with Col 2}
\NormalTok{      df <-}\StringTok{ }\NormalTok{df[, mag_swap] }

      \CommentTok{# 2b. Negate Raw Magnetometer Col3}
\NormalTok{      df[, mag_negate] <-}\StringTok{ }\OperatorTok{-}\NormalTok{df[, mag_negate]}

      \CommentTok{# 2c. Restore original header}
      \KeywordTok{colnames}\NormalTok{(df) <-}\StringTok{ }\NormalTok{o_hdr}
\NormalTok{    \}}
    
    \CommentTok{# 3. Re-order the axes according to position/orientation}
\NormalTok{    df <-}\StringTok{ }\NormalTok{df[, orient_swap] }
    
    \CommentTok{# 4. Restore original header}
    \KeywordTok{colnames}\NormalTok{(df) <-}\StringTok{ }\NormalTok{o_hdr}
    
    \CommentTok{# 5. Negate values for given columns}
\NormalTok{    df[, negate_hdr] <-}\StringTok{ }\OperatorTok{-}\NormalTok{df[, orient_negate]}
    
    \KeywordTok{return}\NormalTok{(df)}
\NormalTok{  \}}
  
  \CommentTok{# store list names}
\NormalTok{  s_list_names <-}\StringTok{ }\KeywordTok{names}\NormalTok{(sensor_list)}
  
  \CommentTok{# apply the swap function to entire list}
\NormalTok{  sensor_list <-}\StringTok{ }\KeywordTok{lapply}\NormalTok{(s_list_names, }
\NormalTok{                        abs_swap, }
                        \DataTypeTok{mag_swap=}\NormalTok{mag_hdr,}
                        \DataTypeTok{mag_negate=}\NormalTok{mag_negate_hdr,}
                        \DataTypeTok{orient_swap=}\NormalTok{orient_hdr, }
                        \DataTypeTok{orient_negate=}\NormalTok{orient_negate_hdr, }
                        \DataTypeTok{sd_list=}\NormalTok{sensor_list)}
  
  \CommentTok{# restore the list names}
  \KeywordTok{names}\NormalTok{(sensor_list) <-}\StringTok{ }\NormalTok{s_list_names}
  
  \CommentTok{# return the adjusted list}
  \KeywordTok{return}\NormalTok{(sensor_list)}
\NormalTok{\}}
\end{Highlighting}
\end{Shaded}


\end{document}
