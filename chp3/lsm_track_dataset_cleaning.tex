\documentclass[]{article}
\usepackage{lmodern}
\usepackage{amssymb,amsmath}
\usepackage{ifxetex,ifluatex}
\usepackage{fixltx2e} % provides \textsubscript
\ifnum 0\ifxetex 1\fi\ifluatex 1\fi=0 % if pdftex
  \usepackage[T1]{fontenc}
  \usepackage[utf8]{inputenc}
\else % if luatex or xelatex
  \ifxetex
    \usepackage{mathspec}
  \else
    \usepackage{fontspec}
  \fi
  \defaultfontfeatures{Ligatures=TeX,Scale=MatchLowercase}
\fi
% use upquote if available, for straight quotes in verbatim environments
\IfFileExists{upquote.sty}{\usepackage{upquote}}{}
% use microtype if available
\IfFileExists{microtype.sty}{%
\usepackage{microtype}
\UseMicrotypeSet[protrusion]{basicmath} % disable protrusion for tt fonts
}{}
\usepackage[margin=1in]{geometry}
\usepackage{hyperref}
\hypersetup{unicode=true,
            pdftitle={MSc Data Science \& Analytics: LSM Track Data Cleaning \& Exploration},
            pdfauthor={Jill Daly},
            pdfborder={0 0 0},
            breaklinks=true}
\urlstyle{same}  % don't use monospace font for urls
\usepackage{color}
\usepackage{fancyvrb}
\newcommand{\VerbBar}{|}
\newcommand{\VERB}{\Verb[commandchars=\\\{\}]}
\DefineVerbatimEnvironment{Highlighting}{Verbatim}{commandchars=\\\{\}}
% Add ',fontsize=\small' for more characters per line
\usepackage{framed}
\definecolor{shadecolor}{RGB}{248,248,248}
\newenvironment{Shaded}{\begin{snugshade}}{\end{snugshade}}
\newcommand{\AlertTok}[1]{\textcolor[rgb]{0.94,0.16,0.16}{#1}}
\newcommand{\AnnotationTok}[1]{\textcolor[rgb]{0.56,0.35,0.01}{\textbf{\textit{#1}}}}
\newcommand{\AttributeTok}[1]{\textcolor[rgb]{0.77,0.63,0.00}{#1}}
\newcommand{\BaseNTok}[1]{\textcolor[rgb]{0.00,0.00,0.81}{#1}}
\newcommand{\BuiltInTok}[1]{#1}
\newcommand{\CharTok}[1]{\textcolor[rgb]{0.31,0.60,0.02}{#1}}
\newcommand{\CommentTok}[1]{\textcolor[rgb]{0.56,0.35,0.01}{\textit{#1}}}
\newcommand{\CommentVarTok}[1]{\textcolor[rgb]{0.56,0.35,0.01}{\textbf{\textit{#1}}}}
\newcommand{\ConstantTok}[1]{\textcolor[rgb]{0.00,0.00,0.00}{#1}}
\newcommand{\ControlFlowTok}[1]{\textcolor[rgb]{0.13,0.29,0.53}{\textbf{#1}}}
\newcommand{\DataTypeTok}[1]{\textcolor[rgb]{0.13,0.29,0.53}{#1}}
\newcommand{\DecValTok}[1]{\textcolor[rgb]{0.00,0.00,0.81}{#1}}
\newcommand{\DocumentationTok}[1]{\textcolor[rgb]{0.56,0.35,0.01}{\textbf{\textit{#1}}}}
\newcommand{\ErrorTok}[1]{\textcolor[rgb]{0.64,0.00,0.00}{\textbf{#1}}}
\newcommand{\ExtensionTok}[1]{#1}
\newcommand{\FloatTok}[1]{\textcolor[rgb]{0.00,0.00,0.81}{#1}}
\newcommand{\FunctionTok}[1]{\textcolor[rgb]{0.00,0.00,0.00}{#1}}
\newcommand{\ImportTok}[1]{#1}
\newcommand{\InformationTok}[1]{\textcolor[rgb]{0.56,0.35,0.01}{\textbf{\textit{#1}}}}
\newcommand{\KeywordTok}[1]{\textcolor[rgb]{0.13,0.29,0.53}{\textbf{#1}}}
\newcommand{\NormalTok}[1]{#1}
\newcommand{\OperatorTok}[1]{\textcolor[rgb]{0.81,0.36,0.00}{\textbf{#1}}}
\newcommand{\OtherTok}[1]{\textcolor[rgb]{0.56,0.35,0.01}{#1}}
\newcommand{\PreprocessorTok}[1]{\textcolor[rgb]{0.56,0.35,0.01}{\textit{#1}}}
\newcommand{\RegionMarkerTok}[1]{#1}
\newcommand{\SpecialCharTok}[1]{\textcolor[rgb]{0.00,0.00,0.00}{#1}}
\newcommand{\SpecialStringTok}[1]{\textcolor[rgb]{0.31,0.60,0.02}{#1}}
\newcommand{\StringTok}[1]{\textcolor[rgb]{0.31,0.60,0.02}{#1}}
\newcommand{\VariableTok}[1]{\textcolor[rgb]{0.00,0.00,0.00}{#1}}
\newcommand{\VerbatimStringTok}[1]{\textcolor[rgb]{0.31,0.60,0.02}{#1}}
\newcommand{\WarningTok}[1]{\textcolor[rgb]{0.56,0.35,0.01}{\textbf{\textit{#1}}}}
\usepackage{graphicx,grffile}
\makeatletter
\def\maxwidth{\ifdim\Gin@nat@width>\linewidth\linewidth\else\Gin@nat@width\fi}
\def\maxheight{\ifdim\Gin@nat@height>\textheight\textheight\else\Gin@nat@height\fi}
\makeatother
% Scale images if necessary, so that they will not overflow the page
% margins by default, and it is still possible to overwrite the defaults
% using explicit options in \includegraphics[width, height, ...]{}
\setkeys{Gin}{width=\maxwidth,height=\maxheight,keepaspectratio}
\IfFileExists{parskip.sty}{%
\usepackage{parskip}
}{% else
\setlength{\parindent}{0pt}
\setlength{\parskip}{6pt plus 2pt minus 1pt}
}
\setlength{\emergencystretch}{3em}  % prevent overfull lines
\providecommand{\tightlist}{%
  \setlength{\itemsep}{0pt}\setlength{\parskip}{0pt}}
\setcounter{secnumdepth}{5}
% Redefines (sub)paragraphs to behave more like sections
\ifx\paragraph\undefined\else
\let\oldparagraph\paragraph
\renewcommand{\paragraph}[1]{\oldparagraph{#1}\mbox{}}
\fi
\ifx\subparagraph\undefined\else
\let\oldsubparagraph\subparagraph
\renewcommand{\subparagraph}[1]{\oldsubparagraph{#1}\mbox{}}
\fi

%%% Use protect on footnotes to avoid problems with footnotes in titles
\let\rmarkdownfootnote\footnote%
\def\footnote{\protect\rmarkdownfootnote}

%%% Change title format to be more compact
\usepackage{titling}

% Create subtitle command for use in maketitle
\newcommand{\subtitle}[1]{
  \posttitle{
    \begin{center}\large#1\end{center}
    }
}

\setlength{\droptitle}{-2em}
  \title{MSc Data Science \& Analytics: LSM Track Data Cleaning \& Exploration}
  \pretitle{\vspace{\droptitle}\centering\huge}
  \posttitle{\par}
  \author{Jill Daly}
  \preauthor{\centering\large\emph}
  \postauthor{\par}
  \predate{\centering\large\emph}
  \postdate{\par}
  \date{15/10/2018}

\usepackage{booktabs}
\usepackage{longtable}
\usepackage{array}
\usepackage{multirow}
\usepackage[table]{xcolor}
\usepackage{wrapfig}
\usepackage{float}
\usepackage{colortbl}
\usepackage{pdflscape}
\usepackage{tabu}
\usepackage{threeparttable}
\usepackage{threeparttablex}
\usepackage[normalem]{ulem}
\usepackage{makecell}

\begin{document}
\maketitle

{
\setcounter{tocdepth}{2}
\tableofcontents
}
\newpage 
\tableofcontents 
\listoffigures
\listoftables
\newpage

\newpage

\hypertarget{introduction}{%
\section{Introduction}\label{introduction}}

This thesis research question stems from research for a separate
Master's thesis project (Catenazzo, 2018), which was completed in
association with the Tyndall National Institute. That project (ibid)
examined data generated from a prototype Wireless Sensor Network (WSN)
for the purpose of designing a Predictive Maintenance (PdM) pipeline for
a manufacturing production environment. The problem of how to accurately
identify useful data observations from experiments, conducted as part of
the Tyndall's prototype, will be addressed by research for this MSc
thesis. Isolation of this useful data will improve cost and accuracy of
the final PdM pipeline for its designated manufacturing pipeline.

The following document is a draft report, which will feed into the final
thesis document. The aim of this draft report is to provide transparency
regarding steps taken while cleaning the data. By outlining the steps
taken, this project aims to meet with open science reproducibility
standards (add ref).

\newpage

\hypertarget{data-capture-experiment---lit-review}{%
\section{Data Capture Experiment - Lit
Review}\label{data-capture-experiment---lit-review}}

Catenazzo's work (ibid) analysed results from two experiments. The first
experiment involved a static application in a laboratory setting. The
results from this experiment led to the second experiment, this time
involving a prototype of a manufacturing production line.

It was while examining results from this second prototype experiment
(ibid), that a gap in the research was identified. A problem emerged of
how to effectively isolate data observations from specific sections of
the track constructed for the prototype experiment. This MSc thesis
seeks a solution to this problem by identifying the state of the vehicle
on the track from the data observations generated by the experiment.

Isolating the data is useful because it represents sensor readings taken
from straight paths along the closed loop prototype. The prototype was
intended to mimic a production environment, however in reality it needed
to transition the vehicle on shuttles between each straight path
section. This was necessary in order to increase the distance traveled
by the vehicle. This shuttle will not exist in production, and the data
generated from the transition to and from the shuttle, as well as the
shuttle readings, need to be omitted from the dataset.

The isolated data observations will be labelled by a supervised learning
classification model, which will be constructed as part of this MSc
thesis. This labelled data will feed into the design and implementation
of a PdM pipeline in a manufacturing production environment.
Successfully omitting the shuttle and transition data, and including the
straight path data will improve the quality of the data, ultimately
improving the accuracy of a PdM pipeline which detects faults of the
system.

Before building the classification model, data cleaning and exploratory
analysis of the data produced by the prototype experiment is required.

** Lookup -- who are MagenMotion. The prototype data capture was
performed using MagneMotion's Linear Synchronous Motor (LSM) driven
assembly system (Ranky, 2007).

The prototype data capture was performed using QuickStick, which is
MagneMotion's patented LSM system. This system propels and controls each
vehicle independently by interacting with a permanent magnet array
mounted to each vehicle (ibid).

{[} TODO -- Show figure from paper/display QuickStick drawing/design {]}

The core technology of an LSM is the linear electric motor. This motor
can convert electrical energy directly into linear motion. QuickStick
LSM is a 1m long, linear motor module with embedded position sensors and
control software. QuickStick controls the velocity and acceleration of
the system. Each node controller can handle 320 QuickSticks on 8 paths.
A production environment could host many node controllers, allowing for
multiple vehicles to be tracked and managed.

For this prototype data capture, the experiment design included 6
QuickStick 1m long modules in a closed loop. These QuickStick modules
were joined to form a ``Straight Path''. There were two straight paths
in the prototype loop, with shuttles bridging the gap between the
straight paths.

{[} TODO -- Insert Path Diagram from Donato's thesis, and reference {]}

\newpage

\hypertarget{second-data-capture-controller-logs}{%
\section{Second Data Capture Controller
Logs}\label{second-data-capture-controller-logs}}

The controler logs document commands issued to the LSM vehicle during
the prototype experiment. In a production environment, there could be
many vehicles on many productions lines, all running in parallel. The
controller would manage each of these vehicles, tracking where each
vehicle is positioned on each line. The secidn data capture prototype
experiment

\begin{Shaded}
\begin{Highlighting}[]
\CommentTok{# These columns are specific to Data Capture May 2018}
\NormalTok{factor_cols <-}\StringTok{ }\KeywordTok{list}\NormalTok{(}\StringTok{"Vehicle"}\NormalTok{, }\StringTok{"Path"}\NormalTok{, }\StringTok{"DstPath"}\NormalTok{, }\StringTok{'Destination'}\NormalTok{, }
                    \StringTok{"Cmd"}\NormalTok{, }\StringTok{"Sig"}\NormalTok{, }\StringTok{"Obs"}\NormalTok{, }\StringTok{"Jam"}\NormalTok{, }\StringTok{"Sus"}\NormalTok{, }\StringTok{"PID"}\NormalTok{)}
\NormalTok{missed_file <-}\StringTok{ }\KeywordTok{paste}\NormalTok{(DC2_LOGS_PATH, }\StringTok{"Acc_1_Vel_4_Test2"}\NormalTok{, }\DataTypeTok{sep =} \StringTok{"/"}\NormalTok{)}
\end{Highlighting}
\end{Shaded}

\begin{Shaded}
\begin{Highlighting}[]
\CommentTok{# Create a list of data frames for the controller log files }
\NormalTok{controller_logs <-}\StringTok{ }\KeywordTok{load_controller_logs}\NormalTok{(DC2_ALL_LOGS, factor_cols, missed_file)}
\end{Highlighting}
\end{Shaded}

\begin{verbatim}
## [1] "error Error in scan(file = file, what = what, sep = sep, quote = quote, dec = dec, : line 204 did not have 12 elements\n  "
\end{verbatim}

The total observation count for the all controller logs is 34972.

Here is a summary of a single log file

\begin{table}[!h]

\caption{\label{tab:summlogfiles}Controller Log File - Summary}
\centering
\resizebox{\linewidth}{!}{
\begin{tabular}[t]{l|l|l|l|l|l|l|l|l|l|l|l|l}
\hline
  & Vehicle & Path & DstPath &    Position &  Destination &    Velocity &  Cmd & Sig & Obs & Jam & Sus & PID\\
\hline
 & 1 :6957 & 1:2653 & 0: 285 & Min.   :-0.1300 & 2.8500 :6153 & Min.   :-0.7076 & 0  : 810 & N:  62 & N:7108 & N:7413 & N:7413 & 0 :7143\\
\hline
 & 51:  73 & 5:4304 & 1:2380 & 1st Qu.: 0.5133 & Idle   : 810 & 1st Qu.: 0.0113 & 177:6603 & Y:7351 & Y: 305 & NA & NA & 15: 270\\
\hline
 & 52: 383 & 6:  73 & 5:4300 & Median : 1.3217 & 0.2000 : 350 & Median : 0.0129 & NA & NA & NA & NA & NA & NA\\
\hline
 & NA & 7: 383 & 6:  69 & Mean   : 1.3613 & 0.1500 :  21 & Mean   : 0.0170 & NA & NA & NA & NA & NA & NA\\
\hline
 & NA & NA & 7: 379 & 3rd Qu.: 2.1382 & 0.3000 :  19 & 3rd Qu.: 0.0137 & NA & NA & NA & NA & NA & NA\\
\hline
 & NA & NA & NA & Max.   : 3.0121 & 2.0000 :  17 & Max.   : 0.7179 & NA & NA & NA & NA & NA & NA\\
\hline
 & NA & NA & NA & NA & (Other):  43 & NA & NA & NA & NA & NA & NA & NA\\
\hline
\end{tabular}}
\end{table}

Here is a summary of contoller log file dimensions for each run:

\begin{table}[!h]

\caption{\label{tab:logfiledetails}Controller Datasets}
\centering
\begin{tabular}[t]{l|r|r}
\hline
df\_name & r\_count & col\_count\\
\hline
Acc\_1\_Vel\_0135\_Test2 & 7413 & 12\\
\hline
Acc\_1\_Vel\_03\_Test2 & 4380 & 12\\
\hline
Acc\_1\_Vel\_06\_Test2 & 2142 & 12\\
\hline
Acc\_1\_Vel\_2\_Test2 & 1379 & 12\\
\hline
Acc\_1\_Vel\_4\_Test2 & 2127 & 18\\
\hline
Acc\_2\_Vel\_06\_Test2 & 1997 & 12\\
\hline
Acc\_2\_Vel\_2\_Test2 & 2771 & 12\\
\hline
Acc\_2\_Vel\_4\_Test2 & 1098 & 12\\
\hline
Acc\_2\_Vel\_4\_Test2\_No\_Pallet & 3587 & 12\\
\hline
Acc\_2\_Vel\_6\_Test2\_No\_Pallet & 8078 & 12\\
\hline
\end{tabular}
\end{table}

This is a sample controller Log file:

\begin{table}[!h]

\caption{\label{tab:strlogfiles}Sample Controller Log File - Structure}
\centering
\resizebox{\linewidth}{!}{
\begin{tabular}[t]{l}
\hline
Log File Structure\\
\hline
'data.frame':   7413 obs. of  12 variables:\\
\hline
\$ Vehicle    : Factor w/ 3 levels "1","51","52": 1 1 1 1 1 1 1 1 1 1 ...\\
\hline
\$ Path       : Factor w/ 4 levels "1","5","6","7": 1 1 1 1 1 1 1 1 1 1 ...\\
\hline
\$ DstPath    : Factor w/ 5 levels "0","1","5","6",..: 1 1 1 1 1 1 1 1 1 1 ...\\
\hline
\$ Position   : num  0.348 0.348 0.348 0.348 0.348 ...\\
\hline
\$ Destination: Factor w/ 9 levels "-2.0000","0.1500",..: 9 9 9 9 9 9 9 9 9 9 ...\\
\hline
\$ Velocity   : num  0e+00 0e+00 0e+00 0e+00 0e+00 -2e-04 -2e-04 2e-04 0e+00 -2e-04 ...\\
\hline
\$ Cmd        : Factor w/ 2 levels "0","177": 1 1 1 1 1 1 1 1 1 1 ...\\
\hline
\$ Sig        : Factor w/ 2 levels "N","Y": 2 2 2 2 2 2 2 2 2 2 ...\\
\hline
\$ Obs        : Factor w/ 2 levels "N","Y": 1 1 1 1 1 1 1 1 1 1 ...\\
\hline
\$ Jam        : Factor w/ 1 level "N": 1 1 1 1 1 1 1 1 1 1 ...\\
\hline
\$ Sus        : Factor w/ 1 level "N": 1 1 1 1 1 1 1 1 1 1 ...\\
\hline
\$ PID        : Factor w/ 2 levels "0","15": 2 2 2 2 2 2 2 2 2 2 ...\\
\hline
\end{tabular}}
\end{table}

\newpage

\hypertarget{controller-logs-missing-data}{%
\subsection{Controller Logs Missing
Data}\label{controller-logs-missing-data}}

There is only one log file with missing data, as is demonstrated with
the data below. This is because the log file
\texttt{Acc\_1\_Vel\_4\_Test2} has a extra columns included, which were
added after the data capture. They can be dicounted as missing data.
These values will form part of the analysis for joining the Sensor and
Controller datasets, so that they are part of a single dataset. This
final dataset will then be presented for the data preprocessing and
model construction in a later section.

\begin{table}[!h]

\caption{\label{tab:log_nas}Controller Missing Data - Summary}
\centering
\begin{tabular}[t]{l}
\hline
Missing Data\\
\hline
Classes 'tbl\_df', 'tbl' and 'data.frame':  1 obs. of  10 variables:\\
\hline
\$ Acc\_1\_Vel\_0135\_Test2       : int 0\\
\hline
\$ Acc\_1\_Vel\_03\_Test2         : int 0\\
\hline
\$ Acc\_1\_Vel\_06\_Test2         : int 0\\
\hline
\$ Acc\_1\_Vel\_2\_Test2          : int 0\\
\hline
\$ Acc\_1\_Vel\_4\_Test2          : int 2127\\
\hline
\$ Acc\_2\_Vel\_06\_Test2         : int 0\\
\hline
\$ Acc\_2\_Vel\_2\_Test2          : int 0\\
\hline
\$ Acc\_2\_Vel\_4\_Test2          : int 0\\
\hline
\$ Acc\_2\_Vel\_4\_Test2\_No\_Pallet: int 0\\
\hline
\$ Acc\_2\_Vel\_6\_Test2\_No\_Pallet: int 0\\
\hline
\end{tabular}
\end{table}

\newpage

\hypertarget{todo---discuss-meaning-of-each-column}{%
\subsubsection{TODO - Discuss meaning of each
column}\label{todo---discuss-meaning-of-each-column}}

\hypertarget{todo---plot-bar-charts-for-logs}{%
\subsubsection{TODO - Plot Bar charts for
Logs}\label{todo---plot-bar-charts-for-logs}}

\hypertarget{todo---add-time-column-to-controller-logs}{%
\subsubsection{TODO - Add Time Column to Controller
Logs}\label{todo---add-time-column-to-controller-logs}}

\hypertarget{todo---create-label-column-from-first-3-columns}{%
\subsubsection{TODO - Create Label Column from first 3
columns}\label{todo---create-label-column-from-first-3-columns}}

\hypertarget{todo---add-section-on-quickstick}{%
\subsubsection{TODO - Add section on
QuickStick}\label{todo---add-section-on-quickstick}}

\newpage

\hypertarget{data-capture-sensor-data-files}{%
\section{Data Capture Sensor Data
Files}\label{data-capture-sensor-data-files}}

TODO - Give description of LSM Vehicle, and Track Give explanation of
each run, and number of sensors Re-iterate that 2nd Data Capture is the
only one of interest here. Explain that there are 3 axes per sensor
Include Image of Sensor Orientation, and also description of MPU925
Sensor

\hypertarget{hardcoded-values}{%
\subsection{Hardcoded Values}\label{hardcoded-values}}

These hardcoded values are only for the May 2018 Data Capture. There are
two main objective to placing these hardcoded items outside of the
functions and logic. Firstly, they provide transparency so that readers
can discern decisions and secondly, because the values are easily
edited, without impacting/breaking the code.

\begin{Shaded}
\begin{Highlighting}[]
\CommentTok{# Pre-defined header for the raw sensor data}
\NormalTok{sensor_header =}\StringTok{ }\KeywordTok{c}\NormalTok{(}\StringTok{"IMU1_X"}\NormalTok{, }\StringTok{"IMU1_Y"}\NormalTok{,   }\StringTok{"IMU1_Z"}\NormalTok{,   }
                  \StringTok{"IMU2_X"}\NormalTok{, }\StringTok{"IMU2_Y"}\NormalTok{,   }\StringTok{"IMU2_Z"}\NormalTok{,}
                  \StringTok{"IMU3_X"}\NormalTok{, }\StringTok{"IMU3_Y"}\NormalTok{,   }\StringTok{"IMU3_Z"}\NormalTok{,}
                  \StringTok{"IMU4_X"}\NormalTok{, }\StringTok{"IMU4_Y"}\NormalTok{,   }\StringTok{"IMU4_Z"}\NormalTok{,}
                  \StringTok{"IMU5_X"}\NormalTok{, }\StringTok{"IMU5_Y"}\NormalTok{,   }\StringTok{"IMU5_Z"}\NormalTok{,}
                  \StringTok{"IMU6_X"}\NormalTok{, }\StringTok{"IMU6_Y"}\NormalTok{,   }\StringTok{"IMU6_Z"}\NormalTok{,}
                  \StringTok{"IMU7_X"}\NormalTok{, }\StringTok{"IMU7_Y"}\NormalTok{,   }\StringTok{"IMU7_Z"}\NormalTok{,}
                  \StringTok{"IMU8_X"}\NormalTok{, }\StringTok{"IMU8_Y"}\NormalTok{,   }\StringTok{"IMU8_Z"}\NormalTok{,}
                  \StringTok{"IMU9_X"}\NormalTok{, }\StringTok{"IMU9_Y"}\NormalTok{,   }\StringTok{"IMU9_Z"}\NormalTok{,   }
                  \StringTok{"IMU10_X"}\NormalTok{, }\StringTok{"IMU10_Y"}\NormalTok{, }\StringTok{"IMU10_Z"}\NormalTok{,}
                  \StringTok{"IMU11_X"}\NormalTok{, }\StringTok{"IMU11_Y"}\NormalTok{, }\StringTok{"IMU11_Z"}\NormalTok{,  }
                  \StringTok{"IMU12_X"}\NormalTok{, }\StringTok{"IMU12_Y"}\NormalTok{, }\StringTok{"IMU12_Z"}\NormalTok{,}
                  \StringTok{"Timestamp"}\NormalTok{, }\StringTok{"Delta"}\NormalTok{)}


\CommentTok{# Magnetometer Data Column 1,2 Swap }
\NormalTok{mag_col_swap =}\StringTok{ }\KeywordTok{c}\NormalTok{(}\StringTok{"IMU1_Y"}\NormalTok{, }\StringTok{"IMU1_X"}\NormalTok{, }\StringTok{"IMU1_Z"}\NormalTok{,  }
                 \StringTok{"IMU2_Y"}\NormalTok{, }\StringTok{"IMU2_X"}\NormalTok{, }\StringTok{"IMU2_Z"}\NormalTok{,}
                 \StringTok{"IMU3_Y"}\NormalTok{, }\StringTok{"IMU3_X"}\NormalTok{, }\StringTok{"IMU3_Z"}\NormalTok{,}
                 \StringTok{"IMU4_Y"}\NormalTok{, }\StringTok{"IMU4_X"}\NormalTok{, }\StringTok{"IMU4_Z"}\NormalTok{,}
                 \StringTok{"IMU5_Y"}\NormalTok{, }\StringTok{"IMU5_X"}\NormalTok{, }\StringTok{"IMU5_Z"}\NormalTok{,}
                 \StringTok{"IMU6_Y"}\NormalTok{, }\StringTok{"IMU6_X"}\NormalTok{, }\StringTok{"IMU6_Z"}\NormalTok{,}
                 \StringTok{"IMU7_Y"}\NormalTok{, }\StringTok{"IMU7_X"}\NormalTok{, }\StringTok{"IMU7_Z"}\NormalTok{,}
                 \StringTok{"IMU8_Y"}\NormalTok{, }\StringTok{"IMU8_X"}\NormalTok{, }\StringTok{"IMU8_Z"}\NormalTok{,}
                 \StringTok{"IMU9_Y"}\NormalTok{, }\StringTok{"IMU9_X"}\NormalTok{, }\StringTok{"IMU9_Z"}\NormalTok{,  }
                 \StringTok{"IMU10_Y"}\NormalTok{, }\StringTok{"IMU10_X"}\NormalTok{, }\StringTok{"IMU10_Z"}\NormalTok{,}
                 \StringTok{"IMU11_Y"}\NormalTok{, }\StringTok{"IMU11_X"}\NormalTok{, }\StringTok{"IMU11_Z"}\NormalTok{,   }
                 \StringTok{"IMU12_Y"}\NormalTok{, }\StringTok{"IMU12_X"}\NormalTok{, }\StringTok{"IMU12_Z"}\NormalTok{,}
                 \StringTok{"Timestamp"}\NormalTok{,   }\StringTok{"Delta"}\NormalTok{)}


\CommentTok{# Magnetometer Negate Col3 (Z Axes)}
\NormalTok{mag_z_negate =}\StringTok{ }\NormalTok{negate_hdr <-}\StringTok{ }\KeywordTok{c}\NormalTok{(}\StringTok{"IMU1_Z"}\NormalTok{, }\StringTok{"IMU2_Z"}\NormalTok{, }\StringTok{"IMU3_Z"}\NormalTok{, }
                               \StringTok{"IMU4_Z"}\NormalTok{, }\StringTok{"IMU5_Z"}\NormalTok{, }\StringTok{"IMU6_Z"}\NormalTok{, }
                               \StringTok{"IMU7_Z"}\NormalTok{, }\StringTok{"IMU8_Z"}\NormalTok{, }\StringTok{"IMU9_Z"}\NormalTok{, }
                               \StringTok{"IMU10_Z"}\NormalTok{, }\StringTok{"IMU11_Z"}\NormalTok{, }\StringTok{"IMU12_Z"}\NormalTok{)}


\CommentTok{# This header swap is to account for orientation of the sensor.}
\NormalTok{swap_position_header =}\StringTok{ }\KeywordTok{c}\NormalTok{(}\StringTok{"IMU1_Y"}\NormalTok{, }\StringTok{"IMU1_X"}\NormalTok{, }\StringTok{"IMU1_Z"}\NormalTok{,  }
                         \StringTok{"IMU2_X"}\NormalTok{, }\StringTok{"IMU2_Y"}\NormalTok{, }\StringTok{"IMU2_Z"}\NormalTok{, }
                         \StringTok{"IMU3_X"}\NormalTok{, }\StringTok{"IMU3_Y"}\NormalTok{, }\StringTok{"IMU3_Z"}\NormalTok{,}
                         \StringTok{"IMU4_X"}\NormalTok{, }\StringTok{"IMU4_Y"}\NormalTok{, }\StringTok{"IMU4_Z"}\NormalTok{,}
                         \StringTok{"IMU5_X"}\NormalTok{, }\StringTok{"IMU5_Y"}\NormalTok{, }\StringTok{"IMU5_Z"}\NormalTok{,}
                         \StringTok{"IMU6_X"}\NormalTok{, }\StringTok{"IMU6_Y"}\NormalTok{, }\StringTok{"IMU6_Z"}\NormalTok{,}
                         \StringTok{"IMU7_Z"}\NormalTok{, }\StringTok{"IMU7_X"}\NormalTok{, }\StringTok{"IMU7_Y"}\NormalTok{,}
                         \StringTok{"IMU8_Y"}\NormalTok{, }\StringTok{"IMU8_X"}\NormalTok{, }\StringTok{"IMU8_Z"}\NormalTok{,}
                         \StringTok{"IMU9_Z"}\NormalTok{, }\StringTok{"IMU9_X"}\NormalTok{, }\StringTok{"IMU9_Y"}\NormalTok{,  }
                         \StringTok{"IMU10_X"}\NormalTok{, }\StringTok{"IMU10_Y"}\NormalTok{, }\StringTok{"IMU10_Z"}\NormalTok{,}
                         \StringTok{"IMU11_X"}\NormalTok{, }\StringTok{"IMU11_Y"}\NormalTok{, }\StringTok{"IMU11_Z"}\NormalTok{,   }
                         \StringTok{"IMU12_X"}\NormalTok{, }\StringTok{"IMU12_Y"}\NormalTok{, }\StringTok{"IMU12_Z"}\NormalTok{,}
                         \StringTok{"Timestamp"}\NormalTok{,   }\StringTok{"Delta"}\NormalTok{)}


\CommentTok{# Axes to negate (using the post converted axis (already adjusted for orientation)}
\NormalTok{negate_hdr <-}\StringTok{ }\KeywordTok{c}\NormalTok{(}\StringTok{"IMU1_Y"}\NormalTok{, }\StringTok{"IMU3_Y"}\NormalTok{, }\StringTok{"IMU3_Z"}\NormalTok{, }
                \StringTok{"IMU8_Z"}\NormalTok{, }\StringTok{"IMU9_Y"}\NormalTok{, }\StringTok{"IMU9_Z"}\NormalTok{, }
                \StringTok{"IMU12_Y"}\NormalTok{, }\StringTok{"IMU12_Z"}\NormalTok{)}


\CommentTok{# There were 9 unique accel/vel settings }
\CommentTok{# (10th run re-used setting, with platform off)}
\NormalTok{run_settings <-}\StringTok{ }\KeywordTok{c}\NormalTok{(}\StringTok{"a_1_v_0_2"}\NormalTok{, }\StringTok{"a_1_v_0_03"}\NormalTok{, }\StringTok{"a_1_v_0_4"}\NormalTok{, }
                  \StringTok{"a_1_v_0_06"}\NormalTok{, }\StringTok{"a_1_v_0_0135"}\NormalTok{, }\StringTok{"a_2_v_0_2"}\NormalTok{, }
                  \StringTok{"a_2_v_0_4"}\NormalTok{, }\StringTok{"a_2_v_0_06"}\NormalTok{, }\StringTok{"a_2_v_0_6"}\NormalTok{ )}


\CommentTok{# IMU Sensor Positions:}
\NormalTok{s_pos_df <-}\StringTok{ }\KeywordTok{data.frame}\NormalTok{(}\KeywordTok{t}\NormalTok{(}\KeywordTok{c}\NormalTok{(}\StringTok{"IMU1_POS"}\NormalTok{=}\StringTok{"FRONT_SURFACE"}\NormalTok{, }
                         \StringTok{"IMU3_POS"}\NormalTok{=}\StringTok{"TOP_SURFACE_WITH_PALETTE"}\NormalTok{,}
                         \StringTok{"IMU7_POS"}\NormalTok{=}\StringTok{"FRONT_SURFACE"}\NormalTok{, }
                         \StringTok{"IMU8_POS"}\NormalTok{=}\StringTok{"FRONT_SURFACE"}\NormalTok{, }
                         \StringTok{"IMU9_POS"}\NormalTok{=}\StringTok{"SIDE_SURFACE"}\NormalTok{, }
                         \StringTok{"IMU11_POS"}\NormalTok{=}\StringTok{"TOP_SURFACE"}\NormalTok{, }
                         \StringTok{"IMU12_POS"}\NormalTok{=}\StringTok{"TOP_SURFACE"}\NormalTok{)))}
\end{Highlighting}
\end{Shaded}

When loading the sensor data, each run dataset is loaded as a dataframe,
and then added to the \texttt{sensor\_data} list. Due to the fact that
there are acceleration and magnetometer sensor measurements, there are
20 dataframes in the \texttt{sensor\_data} list.

\begin{Shaded}
\begin{Highlighting}[]
\CommentTok{# Create a list of data frames for the magnetic and vibration sensor data }
\NormalTok{sensor_data <-}\StringTok{ }\KeywordTok{load_sensor_raw}\NormalTok{(DC2_EXCEL, sensor_header)}
\end{Highlighting}
\end{Shaded}

The total observation count for sensor data is 253708. This is a summary
of the raw sensor data file dimension.

\begin{table}[!h]

\caption{\label{tab:sensorfiledetails}Raw Sensor Datasets}
\centering
\begin{tabular}[t]{l|r|r}
\hline
df\_name & r\_count & col\_count\\
\hline
modular\_a\_1\_v\_0\_2\_acc & 4942 & 38\\
\hline
modular\_a\_1\_v\_0\_2\_mag & 4942 & 38\\
\hline
modular\_a\_1\_v\_0\_03\_acc & 21660 & 38\\
\hline
modular\_a\_1\_v\_0\_03\_mag & 21660 & 38\\
\hline
modular\_a\_1\_v\_0\_4\_acc & 4181 & 38\\
\hline
modular\_a\_1\_v\_0\_4\_mag & 4181 & 38\\
\hline
modular\_a\_1\_v\_0\_06\_acc & 11566 & 38\\
\hline
modular\_a\_1\_v\_0\_06\_mag & 11566 & 38\\
\hline
modular\_a\_1\_v\_0\_0135\_acc & 41559 & 38\\
\hline
modular\_a\_1\_v\_0\_0135\_mag & 41559 & 38\\
\hline
modular\_a\_2\_v\_0\_2\_acc & 12992 & 38\\
\hline
modular\_a\_2\_v\_0\_2\_mag & 12992 & 38\\
\hline
modular\_a\_2\_v\_0\_4\_acc & 4532 & 38\\
\hline
modular\_a\_2\_v\_0\_4\_mag & 4532 & 38\\
\hline
modular\_a\_2\_v\_0\_4\_P\_OFF\_acc & 8582 & 38\\
\hline
modular\_a\_2\_v\_0\_4\_P\_OFF\_mag & 8582 & 38\\
\hline
modular\_a\_2\_v\_0\_06\_acc & 11606 & 38\\
\hline
modular\_a\_2\_v\_0\_06\_mag & 11606 & 38\\
\hline
modular\_a\_2\_v\_0\_6\_P\_OFF\_acc & 5234 & 38\\
\hline
modular\_a\_2\_v\_0\_6\_P\_OFF\_mag & 5234 & 38\\
\hline
\end{tabular}
\end{table}

\newpage

\hypertarget{controller-logs-missing-data-1}{%
\subsection{Controller Logs Missing
Data}\label{controller-logs-missing-data-1}}

For the prototype data experiment, each sensor was positioned in
different location on the vehicle (see section on
\texttt{Hardcoded\ Values} for the various postions for each sensors).
Note, the sensors were not altered between runs.

As a result of different sensor orientation, it is necessary to adjust
the axis columns to their true position in each data frame. Here the
variables from the \texttt{Hardcoded\ Values} section are utilised, when
passed to the \texttt{convert\_axis} function.

\begin{Shaded}
\begin{Highlighting}[]
\CommentTok{# Adjust the XYZ order of axes to account for sensor orientation}
\CommentTok{# Adjust for gravity calibration }
\NormalTok{sensor_data <-}\StringTok{ }\KeywordTok{convert_axis}\NormalTok{(sensor_data, mag_col_swap, mag_z_negate, }
\NormalTok{                            swap_position_header, negate_hdr)}
\end{Highlighting}
\end{Shaded}

\includegraphics{lsm_track_dataset_cleaning_files/figure-latex/aggrsensorfiles-pre_3-1.pdf}
\includegraphics{lsm_track_dataset_cleaning_files/figure-latex/aggrsensorfiles-pre_3-2.pdf}

Here are sample \texttt{aggr} plots displaying the amount of missing
sensor data from the a sample raw dataset. This is due to some sensors
not funtioning during the data capture, due to a number of technical
issues. The second plot represents the rig where the platform was not
present, effectively changing the vehicle payload.

It is evident from the missing data plots that there are a number of
columns that have little or no data. Due to the fact that these are
missing as a result of technical facults, these columns can be dropped.

\begin{Shaded}
\begin{Highlighting}[]
\CommentTok{# Drop Columns & Rows with only NAs, as these are empty variables/rows}
\CommentTok{# due to mechanical issues and should not be included}
\NormalTok{sensor_data <-}\StringTok{ }\KeywordTok{drop_na_only_cols}\NormalTok{(sensor_data, }\FloatTok{.15}\NormalTok{)}
\end{Highlighting}
\end{Shaded}

\begin{figure}
\centering
\includegraphics{lsm_track_dataset_cleaning_files/figure-latex/aggrsensorfiles-post-1.pdf}
\caption{modular\_a\_1\_v\_0\_03\_acc}
\end{figure}

Here is the \texttt{aggr} plot for run modular\_a\_1\_v\_0\_03\_acc
after the un-recorded sensors have been removed. This allows for more
meaningful analysis of missing sensor data, after columns with missing
data above the 85\% threshold were removed.

\newpage

\hypertarget{adding-columns-to-sensor-dataset}{%
\subsection{Adding Columns to Sensor
Dataset}\label{adding-columns-to-sensor-dataset}}

In order to facilitate merging of all acceleration and magnetometer
dataframes, the acceleration, velocity and platform values are added to
each individual dataset.

\begin{Shaded}
\begin{Highlighting}[]
\CommentTok{# Add Velocity and Acceleration columns}
\NormalTok{sensor_data <-}\StringTok{ }\KeywordTok{sensor_vel_accel_list}\NormalTok{(sensor_data, run_settings)}

\CommentTok{# Add Platform column}
\NormalTok{sensor_data <-}\StringTok{ }\KeywordTok{sensor_platform_list}\NormalTok{(sensor_data)}
\end{Highlighting}
\end{Shaded}

This is a summary of the sensor data file dimensions (after dropping
columns with 85\% or more NAs, adding now columns)

\begin{table}[!h]

\caption{\label{tab:sensorfiledetails-post}Post Cleaning Sensor Datasets}
\centering
\begin{tabular}[t]{l|r|r}
\hline
df\_name & r\_count & col\_count\\
\hline
modular\_a\_1\_v\_0\_2\_acc & 4942 & 26\\
\hline
modular\_a\_1\_v\_0\_2\_mag & 4942 & 26\\
\hline
modular\_a\_1\_v\_0\_03\_acc & 21660 & 26\\
\hline
modular\_a\_1\_v\_0\_03\_mag & 21660 & 26\\
\hline
modular\_a\_1\_v\_0\_4\_acc & 4181 & 26\\
\hline
modular\_a\_1\_v\_0\_4\_mag & 4181 & 26\\
\hline
modular\_a\_1\_v\_0\_06\_acc & 11566 & 26\\
\hline
modular\_a\_1\_v\_0\_06\_mag & 11566 & 26\\
\hline
modular\_a\_1\_v\_0\_0135\_acc & 41559 & 26\\
\hline
modular\_a\_1\_v\_0\_0135\_mag & 41559 & 26\\
\hline
modular\_a\_2\_v\_0\_2\_acc & 12992 & 26\\
\hline
modular\_a\_2\_v\_0\_2\_mag & 12992 & 26\\
\hline
modular\_a\_2\_v\_0\_4\_acc & 4532 & 26\\
\hline
modular\_a\_2\_v\_0\_4\_mag & 4532 & 26\\
\hline
modular\_a\_2\_v\_0\_4\_P\_OFF\_acc & 8582 & 23\\
\hline
modular\_a\_2\_v\_0\_4\_P\_OFF\_mag & 8582 & 23\\
\hline
modular\_a\_2\_v\_0\_06\_acc & 11606 & 26\\
\hline
modular\_a\_2\_v\_0\_06\_mag & 11606 & 26\\
\hline
modular\_a\_2\_v\_0\_6\_P\_OFF\_acc & 5234 & 23\\
\hline
modular\_a\_2\_v\_0\_6\_P\_OFF\_mag & 5234 & 23\\
\hline
\end{tabular}
\end{table}

The total observation count for the cleaned sensor data remains 253708.

\newpage

\hypertarget{sensor-axes-summaries-and-boxplots}{%
\subsection{Sensor Axes Summaries and
Boxplots}\label{sensor-axes-summaries-and-boxplots}}

\hypertarget{this-is-a-sample-sensor-data-file}{%
\subsubsection{This is a sample sensor data
file:}\label{this-is-a-sample-sensor-data-file}}

\begin{table}[!h]

\caption{\label{tab:strsdfiles}Sample Sensor Data File - Structure}
\centering
\begin{tabular}[t]{l}
\hline
Data Structure\\
\hline
'data.frame':   21660 obs. of  26 variables:\\
\hline
\$ IMU1\_X      : num  -22.6 -22.9 -26.1 -27 -29.2 ...\\
\hline
\$ IMU1\_Y      : num  19.5 16.5 15.5 17.3 14.9 ...\\
\hline
\$ IMU1\_Z      : num  1046 1048 1043 1044 1047 ...\\
\hline
\$ IMU3\_X      : num  -26.9 -22.9 -24.2 -20.1 -21.6 ...\\
\hline
\$ IMU3\_Y      : num  -22.5 -22.2 -21.7 -21.6 -20.1 ...\\
\hline
\$ IMU3\_Z      : num  1033 1034 1036 1026 1038 ...\\
\hline
\$ IMU7\_X      : num  -2.08 3.66 3.91 1.46 10.62 ...\\
\hline
\$ IMU7\_Y      : num  74.1 78.9 71.8 77.8 73.7 ...\\
\hline
\$ IMU7\_Z      : num  1016 1016 1011 1013 1016 ...\\
\hline
\$ IMU8\_X      : num  200 203 204 202 205 ...\\
\hline
\$ IMU8\_Y      : num  32.3 37.2 30.3 33 31.5 ...\\
\hline
\$ IMU8\_Z      : num  977 980 978 983 977 ...\\
\hline
\$ IMU9\_X      : num  60.1 58.3 63.8 61.8 63.7 ...\\
\hline
\$ IMU9\_Y      : num  -45.3 -39.6 -41.4 -40.5 -40.4 ...\\
\hline
\$ IMU9\_Z      : num  993 994 991 992 991 ...\\
\hline
\$ IMU11\_X     : num  -25.1 -18.8 -15.6 -21.1 -19.8 ...\\
\hline
\$ IMU11\_Y     : num  14.2 12.1 12 12.1 14.2 ...\\
\hline
\$ IMU11\_Z     : num  998 1000 1002 998 1003 ...\\
\hline
\$ IMU12\_X     : num  9.03 9.28 11.96 13.67 13.67 ...\\
\hline
\$ IMU12\_Y     : num  -23.3 -19.4 -24.7 -23.7 -23.7 ...\\
\hline
\$ IMU12\_Z     : num  996 993 998 999 994 ...\\
\hline
\$ Timestamp   : num  0.113 0.123 0.133 0.144 0.163 0.173 0.183 0.193 0.204 0.214 ...\\
\hline
\$ Delta       : num  NA 0.01 0.01 0.011 0.019 ...\\
\hline
\$ velocity    : Factor w/ 1 level "0.03": 1 1 1 1 1 1 1 1 1 1 ...\\
\hline
\$ acceleration: Factor w/ 1 level "1": 1 1 1 1 1 1 1 1 1 1 ...\\
\hline
\$ platform    : Factor w/ 1 level "P\_ON": 1 1 1 1 1 1 1 1 1 1 ...\\
\hline
\end{tabular}
\end{table}

\newpage

\hypertarget{sensor-axes-analysis}{%
\subsubsection{Sensor Axes Analysis}\label{sensor-axes-analysis}}

\hypertarget{magnetometer-x-axes-comparison-for-run-namessensor_data4}{%
\paragraph{\texorpdfstring{Magnetometer X Axes Comparison, for run
\texttt{names(sensor\_data){[}{[}4{]}{]}}}{Magnetometer X Axes Comparison, for run names(sensor\_data){[}{[}4{]}{]}}}\label{magnetometer-x-axes-comparison-for-run-namessensor_data4}}

\begin{table}[!h]

\caption{\label{tab:summsensorfiles-mag-x}X Axis for Acceleration - Summary}
\centering
\resizebox{\linewidth}{!}{
\begin{tabular}[t]{l|l|l|l|l|l|l|l}
\hline
  &     IMU1\_X &     IMU3\_X &     IMU7\_X &     IMU8\_X &     IMU9\_X &    IMU11\_X &    IMU12\_X\\
\hline
 & Min.   :-59.98 & Min.   :-70.771 & Min.   :-802.9 & Min.   :-2849.0 & Min.   :-1273.7 & Min.   :-195.52 & Min.   :-95.51\\
\hline
 & 1st Qu.:458.06 & 1st Qu.:  3.299 & 1st Qu.:-529.6 & 1st Qu.:-1936.0 & 1st Qu.: -764.8 & 1st Qu.:  11.85 & 1st Qu.:138.69\\
\hline
 & Median :479.50 & Median : 10.946 & Median :-517.4 & Median :-1903.2 & Median : -731.0 & Median :  21.44 & Median :149.19\\
\hline
 & Mean   :468.72 & Mean   : 11.925 & Mean   :-497.7 & Mean   :-1856.0 & Mean   : -637.6 & Mean   :  24.37 & Mean   :156.75\\
\hline
 & 3rd Qu.:489.70 & 3rd Qu.: 19.642 & 3rd Qu.:-433.6 & 3rd Qu.:-1670.9 & 3rd Qu.: -449.4 & 3rd Qu.:  34.79 & 3rd Qu.:180.83\\
\hline
 & Max.   :722.41 & Max.   :127.898 & Max.   :-194.3 & Max.   : -841.9 & Max.   :  731.9 & Max.   : 280.99 & Max.   :565.42\\
\hline
 & NA & NA's   :118 & NA & NA & NA & NA's   :6 & NA's   :3\\
\hline
\end{tabular}}
\end{table}

\includegraphics{lsm_track_dataset_cleaning_files/figure-latex/boxplot-mag-x-1.pdf}

\hypertarget{magnetometer-y-axes-comparison-for-run-namessensor_data4}{%
\paragraph{\texorpdfstring{Magnetometer Y Axes Comparison, for run
\texttt{names(sensor\_data){[}{[}4{]}{]}}}{Magnetometer Y Axes Comparison, for run names(sensor\_data){[}{[}4{]}{]}}}\label{magnetometer-y-axes-comparison-for-run-namessensor_data4}}

\begin{table}[!h]

\caption{\label{tab:summsensorfiles-mag-y}Y Axis for Acceleration - Summary}
\centering
\resizebox{\linewidth}{!}{
\begin{tabular}[t]{l|l|l|l|l|l|l|l}
\hline
  &     IMU1\_Y &     IMU3\_Y &     IMU7\_Y &     IMU8\_Y &     IMU9\_Y &    IMU11\_Y &    IMU12\_Y\\
\hline
 & Min.   :-1375.99 & Min.   :-62.5245 & Min.   : 64.17 & Min.   :-440.52 & Min.   :-456.1143 & Min.   :-15.59 & Min.   :-64.17\\
\hline
 & 1st Qu.: -583.41 & 1st Qu.:  0.7497 & 1st Qu.:243.35 & 1st Qu.:-217.11 & 1st Qu.:-288.1827 & 1st Qu.: 44.38 & 1st Qu.: 71.07\\
\hline
 & Median : -556.87 & Median :  6.8972 & Median :256.40 & Median :  -1.20 & Median :  -0.4498 & Median : 50.38 & Median : 78.57\\
\hline
 & Mean   : -568.26 & Mean   :  8.1370 & Mean   :260.29 & Mean   : -67.39 & Mean   : -94.0807 & Mean   : 54.23 & Mean   : 88.26\\
\hline
 & 3rd Qu.: -530.63 & 3rd Qu.: 11.3954 & 3rd Qu.:268.09 & 3rd Qu.:  17.84 & 3rd Qu.:  22.9407 & 3rd Qu.: 65.22 & 3rd Qu.:107.81\\
\hline
 & Max.   :  -34.94 & Max.   : 90.5631 & Max.   :533.63 & Max.   : 185.62 & Max.   : 825.4139 & Max.   :134.50 & Max.   :240.65\\
\hline
 & NA & NA's   :160 & NA & NA's   :72 & NA's   :52 & NA's   :6 & NA\\
\hline
\end{tabular}}
\end{table}

\includegraphics{lsm_track_dataset_cleaning_files/figure-latex/boxplot-mag-y-1.pdf}

\hypertarget{magnetometer-z-axes-comparison-for-run-namessensor_data4}{%
\paragraph{\texorpdfstring{Magnetometer Z Axes Comparison, for run
\texttt{names(sensor\_data){[}{[}4{]}{]}}}{Magnetometer Z Axes Comparison, for run names(sensor\_data){[}{[}4{]}{]}}}\label{magnetometer-z-axes-comparison-for-run-namessensor_data4}}

\begin{table}[!h]

\caption{\label{tab:summsensorfiles-mag-z}z Axis for Acceleration - Summary}
\centering
\resizebox{\linewidth}{!}{
\begin{tabular}[t]{l|l|l|l|l|l|l|l}
\hline
  &     IMU1\_Z &     IMU3\_Z &     IMU7\_Z &     IMU8\_Z &     IMU9\_Z &    IMU11\_Z &    IMU12\_Z\\
\hline
 & Min.   :-3000.6 & Min.   :-86.66 & Min.   :-1762.4 & Min.   : 913.3 & Min.   :-546.2 & Min.   :-1374.9 & Min.   :-1728.9\\
\hline
 & 1st Qu.:-1671.4 & 1st Qu.:-68.22 & 1st Qu.: -912.1 & 1st Qu.:1550.0 & 1st Qu.: 127.4 & 1st Qu.: -807.6 & 1st Qu.: -965.7\\
\hline
 & Median :-1629.1 & Median :-65.22 & Median : -890.6 & Median :2281.0 & Median : 488.4 & Median : -783.4 & Median : -935.9\\
\hline
 & Mean   :-1614.9 & Mean   :-63.76 & Mean   : -902.6 & Mean   :2048.7 & Mean   : 374.0 & Mean   : -801.4 & Mean   : -952.9\\
\hline
 & 3rd Qu.:-1495.9 & 3rd Qu.:-60.73 & 3rd Qu.: -860.6 & 3rd Qu.:2304.1 & 3rd Qu.: 508.3 & 3rd Qu.: -765.4 & 3rd Qu.: -899.8\\
\hline
 & Max.   : -492.1 & Max.   : 33.59 & Max.   : -312.6 & Max.   :3060.6 & Max.   : 908.5 & Max.   : -297.9 & Max.   : -197.0\\
\hline
 & NA & NA & NA & NA & NA's   :3 & NA & NA\\
\hline
\end{tabular}}
\end{table}

\includegraphics{lsm_track_dataset_cleaning_files/figure-latex/boxplot-mag-z-1.pdf}

\hypertarget{acceleration-x-axes-comparison-for-run-namessensor_data3}{%
\paragraph{\texorpdfstring{Acceleration X Axes Comparison, for run
\texttt{names(sensor\_data){[}{[}3{]}{]}}}{Acceleration X Axes Comparison, for run names(sensor\_data){[}{[}3{]}{]}}}\label{acceleration-x-axes-comparison-for-run-namessensor_data3}}

\begin{table}[!h]

\caption{\label{tab:summsensorfiles-acc-x}X Axis for Acceleration - Summary}
\centering
\resizebox{\linewidth}{!}{
\begin{tabular}[t]{l|l|l|l|l|l|l|l}
\hline
  &     IMU1\_X &     IMU3\_X &     IMU7\_X &     IMU8\_X &     IMU9\_X &    IMU11\_X &    IMU12\_X\\
\hline
 & Min.   :-1161.01 & Min.   :-1215.09 & Min.   :-502.075 & Min.   :-273.1 & Min.   :-470.46 & Min.   :-422.36 & Min.   :-603.15\\
\hline
 & 1st Qu.:  -29.66 & 1st Qu.:  -18.92 & 1st Qu.:   6.470 & 1st Qu.: 202.3 & 1st Qu.:  57.62 & 1st Qu.: -23.68 & 1st Qu.:  13.06\\
\hline
 & Median :  -27.10 & Median :  -15.87 & Median :   9.399 & Median : 205.0 & Median :  60.67 & Median : -20.63 & Median :  15.62\\
\hline
 & Mean   :  -27.09 & Mean   :  -16.02 & Mean   :   9.308 & Mean   : 205.0 & Mean   :  60.72 & Mean   : -20.75 & Mean   :  15.67\\
\hline
 & 3rd Qu.:  -24.54 & 3rd Qu.:  -12.94 & 3rd Qu.:  12.329 & 3rd Qu.: 207.6 & 3rd Qu.:  63.96 & 3rd Qu.: -17.67 & 3rd Qu.:  18.31\\
\hline
 & Max.   :  501.59 & Max.   :  671.88 & Max.   : 456.177 & Max.   : 926.8 & Max.   : 987.67 & Max.   : 420.41 & Max.   : 490.84\\
\hline
 & NA's   :1 & NA's   :6 & NA's   :37 & NA & NA & NA's   :3 & NA's   :7\\
\hline
\end{tabular}}
\end{table}

\includegraphics{lsm_track_dataset_cleaning_files/figure-latex/boxplot-acc-x-1.pdf}

\hypertarget{acceleration-y-axes-comparison-for-run-namessensor_data3}{%
\paragraph{\texorpdfstring{Acceleration Y Axes Comparison, for run
\texttt{names(sensor\_data){[}{[}3{]}{]}}}{Acceleration Y Axes Comparison, for run names(sensor\_data){[}{[}3{]}{]}}}\label{acceleration-y-axes-comparison-for-run-namessensor_data3}}

\begin{table}[!h]

\caption{\label{tab:summsensorfiles-acc-y}Y Axis for Acceleration - Summary}
\centering
\resizebox{\linewidth}{!}{
\begin{tabular}[t]{l|l|l|l|l|l|l|l}
\hline
  &     IMU1\_Y &     IMU3\_Y &     IMU7\_Y &     IMU8\_Y &     IMU9\_Y &    IMU11\_Y &    IMU12\_Y\\
\hline
 & Min.   :-1545.90 & Min.   :-794.56 & Min.   :-464.48 & Min.   :-715.82 & Min.   :-1028.69 & Min.   :-750.00 & Min.   :-692.14\\
\hline
 & 1st Qu.:   15.62 & 1st Qu.: -27.10 & 1st Qu.:  72.39 & 1st Qu.:  31.62 & 1st Qu.:  -44.19 & 1st Qu.:  12.57 & 1st Qu.: -23.56\\
\hline
 & Median :   17.70 & Median : -25.15 & Median :  74.46 & Median :  33.69 & Median :  -41.87 & Median :  14.40 & Median : -21.73\\
\hline
 & Mean   :   17.70 & Mean   : -25.22 & Mean   :  74.47 & Mean   :  33.78 & Mean   :  -41.78 & Mean   :  14.47 & Mean   : -21.65\\
\hline
 & 3rd Qu.:   19.65 & 3rd Qu.: -23.07 & 3rd Qu.:  76.66 & 3rd Qu.:  36.13 & 3rd Qu.:  -39.18 & 3rd Qu.:  16.60 & 3rd Qu.: -19.53\\
\hline
 & Max.   :  627.44 & Max.   :2255.62 & Max.   :1042.11 & Max.   :1269.04 & Max.   : 2971.92 & Max.   : 673.34 & Max.   : 776.61\\
\hline
 & NA & NA & NA & NA & NA & NA's   :1 & NA's   :1\\
\hline
\end{tabular}}
\end{table}

\includegraphics{lsm_track_dataset_cleaning_files/figure-latex/boxplot-acc-y-1.pdf}

\hypertarget{acceleration-z-axes-comparison-for-run-namessensor_data3}{%
\paragraph{\texorpdfstring{Acceleration Z Axes Comparison, for run
\texttt{names(sensor\_data){[}{[}3{]}{]}}}{Acceleration Z Axes Comparison, for run names(sensor\_data){[}{[}3{]}{]}}}\label{acceleration-z-axes-comparison-for-run-namessensor_data3}}

\begin{table}[!h]

\caption{\label{tab:summsensorfiles-acc-z}Z Axis for Acceleration - Summary}
\centering
\resizebox{\linewidth}{!}{
\begin{tabular}[t]{l|l|l|l|l|l|l|l}
\hline
  &     IMU1\_Z &     IMU3\_Z &     IMU7\_Z &     IMU8\_Z &     IMU9\_Z &    IMU11\_Z &    IMU12\_Z\\
\hline
 & Min.   :-273.1 & Min.   :-481.3 & Min.   :-438.7 & Min.   :  15.38 & Min.   :-988.4 & Min.   :-346.2 & Min.   :-234.4\\
\hline
 & 1st Qu.:1044.3 & 1st Qu.:1031.2 & 1st Qu.:1012.5 & 1st Qu.: 975.22 & 1st Qu.: 991.2 & 1st Qu.: 994.5 & 1st Qu.: 991.9\\
\hline
 & Median :1046.4 & Median :1033.6 & Median :1013.8 & Median : 977.54 & Median : 993.0 & Median : 996.8 & Median : 993.8\\
\hline
 & Mean   :1046.4 & Mean   :1033.6 & Mean   :1013.7 & Mean   : 977.57 & Mean   : 993.1 & Mean   : 996.8 & Mean   : 993.6\\
\hline
 & 3rd Qu.:1048.5 & 3rd Qu.:1035.8 & 3rd Qu.:1015.0 & 3rd Qu.: 979.74 & 3rd Qu.: 994.9 & 3rd Qu.: 999.3 & 3rd Qu.: 995.7\\
\hline
 & Max.   :2042.6 & Max.   :3607.3 & Max.   :1839.8 & Max.   :2264.28 & Max.   :2148.8 & Max.   :1776.2 & Max.   :1400.9\\
\hline
\end{tabular}}
\end{table}

\includegraphics{lsm_track_dataset_cleaning_files/figure-latex/boxplot-acc-z-1.pdf}

\newpage

\hypertarget{sensor-axes-boxplots-for-run-namessensor_data4-and-namessensor_data3}{%
\subsubsection{\texorpdfstring{Sensor Axes Boxplots, for run
\texttt{names(sensor\_data){[}{[}4{]}{]}} and
\texttt{names(sensor\_data){[}{[}3{]}{]}}}{Sensor Axes Boxplots, for run names(sensor\_data){[}{[}4{]}{]} and names(sensor\_data){[}{[}3{]}{]}}}\label{sensor-axes-boxplots-for-run-namessensor_data4-and-namessensor_data3}}

\includegraphics{lsm_track_dataset_cleaning_files/figure-latex/corrplot-mag-x-1.pdf}

\begin{verbatim}
## null device 
##           1
\end{verbatim}

\includegraphics{lsm_track_dataset_cleaning_files/figure-latex/corrplot-mag-y-1.pdf}

\begin{verbatim}
## null device 
##           1
\end{verbatim}

\includegraphics{lsm_track_dataset_cleaning_files/figure-latex/corrplot-mag-z-1.pdf}

\begin{verbatim}
## null device 
##           1
\end{verbatim}

\includegraphics{lsm_track_dataset_cleaning_files/figure-latex/corrplot-acc-x-1.pdf}

\begin{verbatim}
## null device 
##           1
\end{verbatim}

\includegraphics{lsm_track_dataset_cleaning_files/figure-latex/corrplot-acc-y-1.pdf}

\begin{verbatim}
## null device 
##           1
\end{verbatim}

\includegraphics{lsm_track_dataset_cleaning_files/figure-latex/corrplot-acc-z-1.pdf}

\begin{verbatim}
## null device 
##           1
\end{verbatim}

\hypertarget{todo---anova-models}{%
\subsubsection{TODO - Anova models}\label{todo---anova-models}}

\hypertarget{todo---give-details-on-each-of-the-columns}{%
\subsubsection{TODO - Give details on each of the
Columns}\label{todo---give-details-on-each-of-the-columns}}

\hypertarget{todo---examine-outliers}{%
\subsubsection{TODO - Examine Outliers}\label{todo---examine-outliers}}

\hypertarget{todo---examine-missing-data}{%
\subsubsection{TODO - Examine Missing
Data}\label{todo---examine-missing-data}}

\hypertarget{todo---examine-distribution}{%
\subsubsection{TODO - Examine
Distribution}\label{todo---examine-distribution}}


\end{document}
