\documentclass[]{article}
\usepackage{lmodern}
\usepackage{amssymb,amsmath}
\usepackage{ifxetex,ifluatex}
\usepackage{fixltx2e} % provides \textsubscript
\ifnum 0\ifxetex 1\fi\ifluatex 1\fi=0 % if pdftex
  \usepackage[T1]{fontenc}
  \usepackage[utf8]{inputenc}
\else % if luatex or xelatex
  \ifxetex
    \usepackage{mathspec}
  \else
    \usepackage{fontspec}
  \fi
  \defaultfontfeatures{Ligatures=TeX,Scale=MatchLowercase}
\fi
% use upquote if available, for straight quotes in verbatim environments
\IfFileExists{upquote.sty}{\usepackage{upquote}}{}
% use microtype if available
\IfFileExists{microtype.sty}{%
\usepackage{microtype}
\UseMicrotypeSet[protrusion]{basicmath} % disable protrusion for tt fonts
}{}
\usepackage[margin=1in]{geometry}
\usepackage{hyperref}
\hypersetup{unicode=true,
            pdftitle={Chp 4 Results - Random Forest},
            pdfauthor={Jill Daly},
            pdfborder={0 0 0},
            breaklinks=true}
\urlstyle{same}  % don't use monospace font for urls
\usepackage{graphicx,grffile}
\makeatletter
\def\maxwidth{\ifdim\Gin@nat@width>\linewidth\linewidth\else\Gin@nat@width\fi}
\def\maxheight{\ifdim\Gin@nat@height>\textheight\textheight\else\Gin@nat@height\fi}
\makeatother
% Scale images if necessary, so that they will not overflow the page
% margins by default, and it is still possible to overwrite the defaults
% using explicit options in \includegraphics[width, height, ...]{}
\setkeys{Gin}{width=\maxwidth,height=\maxheight,keepaspectratio}
\IfFileExists{parskip.sty}{%
\usepackage{parskip}
}{% else
\setlength{\parindent}{0pt}
\setlength{\parskip}{6pt plus 2pt minus 1pt}
}
\setlength{\emergencystretch}{3em}  % prevent overfull lines
\providecommand{\tightlist}{%
  \setlength{\itemsep}{0pt}\setlength{\parskip}{0pt}}
\setcounter{secnumdepth}{0}
% Redefines (sub)paragraphs to behave more like sections
\ifx\paragraph\undefined\else
\let\oldparagraph\paragraph
\renewcommand{\paragraph}[1]{\oldparagraph{#1}\mbox{}}
\fi
\ifx\subparagraph\undefined\else
\let\oldsubparagraph\subparagraph
\renewcommand{\subparagraph}[1]{\oldsubparagraph{#1}\mbox{}}
\fi

%%% Use protect on footnotes to avoid problems with footnotes in titles
\let\rmarkdownfootnote\footnote%
\def\footnote{\protect\rmarkdownfootnote}

%%% Change title format to be more compact
\usepackage{titling}

% Create subtitle command for use in maketitle
\newcommand{\subtitle}[1]{
  \posttitle{
    \begin{center}\large#1\end{center}
    }
}

\setlength{\droptitle}{-2em}

  \title{Chp 4 Results - Random Forest}
    \pretitle{\vspace{\droptitle}\centering\huge}
  \posttitle{\par}
    \author{Jill Daly}
    \preauthor{\centering\large\emph}
  \postauthor{\par}
      \predate{\centering\large\emph}
  \postdate{\par}
    \date{17 December, 2018}

\usepackage{booktabs}
\usepackage{longtable}
\usepackage{array}
\usepackage{multirow}
\usepackage[table]{xcolor}
\usepackage{wrapfig}
\usepackage{float}
\usepackage{colortbl}
\usepackage{pdflscape}
\usepackage{tabu}
\usepackage{threeparttable}
\usepackage{threeparttablex}
\usepackage[normalem]{ulem}
\usepackage{makecell}

\begin{document}
\maketitle

\begin{figure}
\centering
\includegraphics{3-CHP4_RF_Results-Sensors_files/figure-latex/3g-stacked-sensor-rf-plot-1.pdf}
\caption{Stacked Sensor - RF Model}
\end{figure}

\begin{table}[!h]

\caption{\label{tab:3g-stacked-sensor-rf-params}Stacked Sensor - RF Training Model Results}
\centering
\begin{tabular}[t]{rrlrrrrrrrrrrrrrrrrrrrrrrrrrrrr}
\toprule
mtry & min.node.size & splitrule & logLoss & AUC & prAUC & Accuracy & Kappa & Mean\_F1 & Mean\_Sensitivity & Mean\_Specificity & Mean\_Pos\_Pred\_Value & Mean\_Neg\_Pred\_Value & Mean\_Precision & Mean\_Recall & Mean\_Detection\_Rate & Mean\_Balanced\_Accuracy & logLossSD & AUCSD & prAUCSD & AccuracySD & KappaSD & Mean\_F1SD & Mean\_SensitivitySD & Mean\_SpecificitySD & Mean\_Pos\_Pred\_ValueSD & Mean\_Neg\_Pred\_ValueSD & Mean\_PrecisionSD & Mean\_RecallSD & Mean\_Detection\_RateSD & Mean\_Balanced\_AccuracySD\\
\midrule
2 & 1 & gini & 1.838943 & 0.8334363 & 0.5653817 & 0.4068072 & 0.2267510 & 0.3675995 & 0.4548819 & 0.8173321 & 0.5018577 & 0.8131859 & 0.5018577 & 0.4548819 & 0.1017018 & 0.6361070 & 0.1982157 & 0.0092096 & 0.0233231 & 0.0240565 & 0.0220866 & 0.0181314 & 0.0234178 & 0.0055042 & 0.0179769 & 0.0055802 & 0.0179769 & 0.0234178 & 0.0060141 & 0.0141863\\
2 & 1 & extratrees & 1.223643 & 0.8759130 & 0.6546959 & 0.4506805 & 0.2702200 & 0.3985333 & 0.4840740 & 0.8285950 & 0.5617315 & 0.8263240 & 0.5617315 & 0.4840740 & 0.1126701 & 0.6563345 & 0.0744963 & 0.0147044 & 0.0279432 & 0.0202826 & 0.0236128 & 0.0187803 & 0.0201941 & 0.0063762 & 0.0273418 & 0.0063678 & 0.0273418 & 0.0201941 & 0.0050707 & 0.0132153\\
3 & 1 & gini & 10.182192 & 0.7059564 & 0.4171006 & 0.3927303 & 0.2066889 & 0.3530956 & 0.4407699 & 0.8110542 & 0.4765207 & 0.8071311 & 0.4765207 & 0.4407699 & 0.0981826 & 0.6259120 & 1.1878857 & 0.0165781 & 0.0184217 & 0.0256037 & 0.0212204 & 0.0191120 & 0.0221908 & 0.0051749 & 0.0214142 & 0.0052291 & 0.0214142 & 0.0221908 & 0.0064009 & 0.0132652\\
3 & 1 & extratrees & 1.126891 & 0.8902453 & 0.6866989 & 0.4561525 & 0.2766039 & 0.4151573 & 0.4988342 & 0.8300726 & 0.5614583 & 0.8271002 & 0.5614583 & 0.4988342 & 0.1140381 & 0.6644534 & 0.1151863 & 0.0245392 & 0.0532460 & 0.0299806 & 0.0368524 & 0.0391858 & 0.0432755 & 0.0098045 & 0.0326423 & 0.0094854 & 0.0326423 & 0.0432755 & 0.0074951 & 0.0263520\\
\bottomrule
\end{tabular}
\end{table}

\begin{table}[!h]

\caption{\label{tab:3g-stacked-sensor-rf-params}Stacked Sensor - RF Model Tuning Params}
\centering
\begin{tabular}[t]{lll}
\toprule
parameter & class & label\\
\midrule
mtry & numeric & \#Randomly Selected Predictors\\
splitrule & character & Splitting Rule\\
min.node.size & numeric & Minimal Node Size\\
\bottomrule
\end{tabular}
\end{table}

\begin{table}[!h]

\caption{\label{tab:3g-stacked-sensor-rf-params}Stacked Sensor - RF Best Tuned Model from Caret}
\centering
\begin{tabular}[t]{lrlr}
\toprule
  & mtry & splitrule & min.node.size\\
\midrule
4 & 3 & extratrees & 1\\
\bottomrule
\end{tabular}
\end{table}

\begin{table}[!h]

\caption{\label{tab:3g-stacked-sensor-rf-params}Stacked Sensor - RF Model Coefficients}
\centering
\begin{tabular}[t]{l}
\toprule
x\\
\midrule
Timestamp\\
X\_AXIS\\
Y\_AXIS\\
\bottomrule
\end{tabular}
\end{table}

\begin{verbatim}
## Ranger result
## 
## Call:
##  ranger::ranger(dependent.variable.name = ".outcome", data = x,      mtry = param$mtry, min.node.size = param$min.node.size, splitrule = as.character(param$splitrule),      write.forest = TRUE, probability = classProbs, ...) 
## 
## Type:                             Probability estimation 
## Number of trees:                  500 
## Sample size:                      59304 
## Number of independent variables:  3 
## Mtry:                             3 
## Target node size:                 1 
## Variable importance mode:         none 
## Splitrule:                        extratrees 
## OOB prediction error (Brier s.):  3.630582e-05
\end{verbatim}

\begin{verbatim}
## Random Forest 
## 
## 59304 samples
##     3 predictor
##     4 classes: 'PATH_IDLE', 'PATH_MOVING', 'PATH_TRANSITION', 'SHUTTLE' 
## 
## Pre-processing: nearest neighbor imputation (3), centered (3), scaled (3) 
## Resampling: Bootstrapped (10 reps) 
## Summary of sample sizes: 989, 988, 988, 988, 988, 988, ... 
## Resampling results across tuning parameters:
## 
##   mtry  splitrule   logLoss    AUC        prAUC      Accuracy   Kappa    
##   2     gini         1.838943  0.8334363  0.5653817  0.4068072  0.2267510
##   2     extratrees   1.223643  0.8759130  0.6546959  0.4506805  0.2702200
##   3     gini        10.182192  0.7059564  0.4171006  0.3927303  0.2066889
##   3     extratrees   1.126891  0.8902453  0.6866989  0.4561525  0.2766039
##   Mean_F1    Mean_Sensitivity  Mean_Specificity  Mean_Pos_Pred_Value
##   0.3675995  0.4548819         0.8173321         0.5018577          
##   0.3985333  0.4840740         0.8285950         0.5617315          
##   0.3530956  0.4407699         0.8110542         0.4765207          
##   0.4151573  0.4988342         0.8300726         0.5614583          
##   Mean_Neg_Pred_Value  Mean_Precision  Mean_Recall  Mean_Detection_Rate
##   0.8131859            0.5018577       0.4548819    0.10170179         
##   0.8263240            0.5617315       0.4840740    0.11267013         
##   0.8071311            0.4765207       0.4407699    0.09818257         
##   0.8271002            0.5614583       0.4988342    0.11403813         
##   Mean_Balanced_Accuracy
##   0.6361070             
##   0.6563345             
##   0.6259120             
##   0.6644534             
## 
## Tuning parameter 'min.node.size' was held constant at a value of 1
## logLoss was used to select the optimal model using the smallest value.
## The final values used for the model were mtry = 3, splitrule =
##  extratrees and min.node.size = 1.
\end{verbatim}

\begin{table}[!h]

\caption{\label{tab:3g-stacked-sensor-rf-results}Stacked Sensor - RF - Validation Accuracy}
\centering
\begin{tabular}[t]{lr}
\toprule
  & x\\
\midrule
Accuracy & 1\\
\bottomrule
\end{tabular}
\end{table}

\begin{table}[!h]

\caption{\label{tab:3g-stacked-sensor-rf-results}Stacked Sensor - RF Validation Metrics}
\centering
\begin{tabular}[t]{lrrl}
\toprule
  & Sensitivity & Specificity & MultiClassAUC\\
\midrule
PATH\_IDLE & 1 & 1 & 1\\
PATH\_MOVING & 1 & 1 & 1\\
PATH\_TRANSITION & 1 & 1 & 1\\
SHUTTLE & 1 & 1 & 1\\
\bottomrule
\end{tabular}
\end{table}

\begin{table}[!h]

\caption{\label{tab:3g-stacked-sensor-rf-results}Stacked Sensor - RFF Validation Confusion Matrix}
\centering
\begin{tabular}[t]{lrrrr}
\toprule
  & PATH\_IDLE & PATH\_MOVING & PATH\_TRANSITION & SHUTTLE\\
\midrule
PATH\_IDLE & 4524 & 0 & 0 & 0\\
PATH\_MOVING & 0 & 3712 & 0 & 0\\
PATH\_TRANSITION & 0 & 0 & 542 & 0\\
SHUTTLE & 0 & 0 & 0 & 1106\\
\bottomrule
\end{tabular}
\end{table}

\begin{table}[!h]

\caption{\label{tab:3g-stacked-sensor-rf-results}Stacked Sensor - RFF Vaidation LogLoss}
\centering
\begin{tabular}[t]{r}
\toprule
x\\
\midrule
NA\\
\bottomrule
\end{tabular}
\end{table}

\begin{figure}
\centering
\includegraphics{3-CHP4_RF_Results-Sensors_files/figure-latex/3g-stacked-sensor-rf-roc-1.pdf}
\caption{Stacked Sensor - RF ROC - using 4 Binary ROC Curves}
\end{figure}


\end{document}
