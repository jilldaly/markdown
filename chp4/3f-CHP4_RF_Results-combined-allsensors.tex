\documentclass[]{article}
\usepackage{lmodern}
\usepackage{amssymb,amsmath}
\usepackage{ifxetex,ifluatex}
\usepackage{fixltx2e} % provides \textsubscript
\ifnum 0\ifxetex 1\fi\ifluatex 1\fi=0 % if pdftex
  \usepackage[T1]{fontenc}
  \usepackage[utf8]{inputenc}
\else % if luatex or xelatex
  \ifxetex
    \usepackage{mathspec}
  \else
    \usepackage{fontspec}
  \fi
  \defaultfontfeatures{Ligatures=TeX,Scale=MatchLowercase}
\fi
% use upquote if available, for straight quotes in verbatim environments
\IfFileExists{upquote.sty}{\usepackage{upquote}}{}
% use microtype if available
\IfFileExists{microtype.sty}{%
\usepackage{microtype}
\UseMicrotypeSet[protrusion]{basicmath} % disable protrusion for tt fonts
}{}
\usepackage[margin=1in]{geometry}
\usepackage{hyperref}
\hypersetup{unicode=true,
            pdftitle={Chp 4 Results - Random Forest},
            pdfauthor={Jill Daly},
            pdfborder={0 0 0},
            breaklinks=true}
\urlstyle{same}  % don't use monospace font for urls
\usepackage{graphicx,grffile}
\makeatletter
\def\maxwidth{\ifdim\Gin@nat@width>\linewidth\linewidth\else\Gin@nat@width\fi}
\def\maxheight{\ifdim\Gin@nat@height>\textheight\textheight\else\Gin@nat@height\fi}
\makeatother
% Scale images if necessary, so that they will not overflow the page
% margins by default, and it is still possible to overwrite the defaults
% using explicit options in \includegraphics[width, height, ...]{}
\setkeys{Gin}{width=\maxwidth,height=\maxheight,keepaspectratio}
\IfFileExists{parskip.sty}{%
\usepackage{parskip}
}{% else
\setlength{\parindent}{0pt}
\setlength{\parskip}{6pt plus 2pt minus 1pt}
}
\setlength{\emergencystretch}{3em}  % prevent overfull lines
\providecommand{\tightlist}{%
  \setlength{\itemsep}{0pt}\setlength{\parskip}{0pt}}
\setcounter{secnumdepth}{0}
% Redefines (sub)paragraphs to behave more like sections
\ifx\paragraph\undefined\else
\let\oldparagraph\paragraph
\renewcommand{\paragraph}[1]{\oldparagraph{#1}\mbox{}}
\fi
\ifx\subparagraph\undefined\else
\let\oldsubparagraph\subparagraph
\renewcommand{\subparagraph}[1]{\oldsubparagraph{#1}\mbox{}}
\fi

%%% Use protect on footnotes to avoid problems with footnotes in titles
\let\rmarkdownfootnote\footnote%
\def\footnote{\protect\rmarkdownfootnote}

%%% Change title format to be more compact
\usepackage{titling}

% Create subtitle command for use in maketitle
\newcommand{\subtitle}[1]{
  \posttitle{
    \begin{center}\large#1\end{center}
    }
}

\setlength{\droptitle}{-2em}

  \title{Chp 4 Results - Random Forest}
    \pretitle{\vspace{\droptitle}\centering\huge}
  \posttitle{\par}
    \author{Jill Daly}
    \preauthor{\centering\large\emph}
  \postauthor{\par}
      \predate{\centering\large\emph}
  \postdate{\par}
    \date{07 December, 2018}

\usepackage{booktabs}
\usepackage{longtable}
\usepackage{array}
\usepackage{multirow}
\usepackage[table]{xcolor}
\usepackage{wrapfig}
\usepackage{float}
\usepackage{colortbl}
\usepackage{pdflscape}
\usepackage{tabu}
\usepackage{threeparttable}
\usepackage{threeparttablex}
\usepackage[normalem]{ulem}
\usepackage{makecell}

\begin{document}
\maketitle

\begin{figure}
\centering
\includegraphics{CHP4_RF_Results_files/figure-latex/sensor-rf-plot-1.pdf}
\caption{Sensor (IMU 1,3,7,8,9,11) Model}
\end{figure}

\begin{table}[!h]

\caption{\label{tab:sensor-rf-params}Sensor (IMU 1,3,7,8,9,11) RF Training Model Results}
\centering
\begin{tabular}[t]{rrlrrrrrrrrrrrrrrrrrrrrrrrrrrrr}
\toprule
mtry & min.node.size & splitrule & logLoss & AUC & prAUC & Accuracy & Kappa & Mean\_F1 & Mean\_Sensitivity & Mean\_Specificity & Mean\_Pos\_Pred\_Value & Mean\_Neg\_Pred\_Value & Mean\_Precision & Mean\_Recall & Mean\_Detection\_Rate & Mean\_Balanced\_Accuracy & logLossSD & AUCSD & prAUCSD & AccuracySD & KappaSD & Mean\_F1SD & Mean\_SensitivitySD & Mean\_SpecificitySD & Mean\_Pos\_Pred\_ValueSD & Mean\_Neg\_Pred\_ValueSD & Mean\_PrecisionSD & Mean\_RecallSD & Mean\_Detection\_RateSD & Mean\_Balanced\_AccuracySD\\
\midrule
2 & 1 & gini & 2.244775 & 0.8106723 & 0.5506284 & 0.3497281 & 0.1713700 & 0.3057264 & 0.4023020 & 0.8017163 & 0.5101218 & 0.8025946 & 0.5101218 & 0.4023020 & 0.0874320 & 0.6020092 & 0.6222404 & 0.0148359 & 0.0266170 & 0.0215225 & 0.0138441 & 0.0182792 & 0.0158201 & 0.0031711 & 0.0305015 & 0.0025550 & 0.0305015 & 0.0158201 & 0.0053806 & 0.0093097\\
2 & 1 & extratrees & 1.575715 & 0.8355501 & 0.5794596 & 0.3725607 & 0.2007435 & 0.3163320 & 0.4176647 & 0.8112342 & 0.5745317 & 0.8123653 & 0.5745317 & 0.4176647 & 0.0931402 & 0.6144494 & 0.0732806 & 0.0109946 & 0.0147103 & 0.0218886 & 0.0192036 & 0.0141590 & 0.0140930 & 0.0049769 & 0.0396106 & 0.0047970 & 0.0396106 & 0.0140930 & 0.0054722 & 0.0094619\\
3 & 1 & gini & 3.059662 & 0.7818548 & 0.5132026 & 0.3467152 & 0.1661478 & 0.3120710 & 0.4037873 & 0.7999871 & 0.4884279 & 0.7995287 & 0.4884279 & 0.4037873 & 0.0866788 & 0.6018872 & 1.0263044 & 0.0178119 & 0.0283426 & 0.0272270 & 0.0190933 & 0.0224727 & 0.0196976 & 0.0048136 & 0.0210823 & 0.0041236 & 0.0210823 & 0.0196976 & 0.0068068 & 0.0119348\\
3 & 1 & extratrees & 1.382679 & 0.8680137 & 0.6345818 & 0.4104649 & 0.2375047 & 0.3550340 & 0.4521986 & 0.8208922 & 0.5865778 & 0.8213790 & 0.5865778 & 0.4521986 & 0.1026162 & 0.6365454 & 0.1063412 & 0.0170313 & 0.0246167 & 0.0136507 & 0.0128885 & 0.0136704 & 0.0125759 & 0.0035174 & 0.0312706 & 0.0033939 & 0.0312706 & 0.0125759 & 0.0034127 & 0.0079019\\
4 & 1 & gini & 10.645753 & 0.6912447 & 0.4210159 & 0.3439510 & 0.1625890 & 0.3123221 & 0.4096034 & 0.7985001 & 0.4730115 & 0.7975549 & 0.4730115 & 0.4096034 & 0.0859878 & 0.6040517 & 1.5906150 & 0.0175205 & 0.0204494 & 0.0327554 & 0.0266037 & 0.0264623 & 0.0302968 & 0.0069521 & 0.0205276 & 0.0059076 & 0.0205276 & 0.0302968 & 0.0081889 & 0.0184343\\
4 & 1 & extratrees & 1.294548 & 0.8804000 & 0.6628977 & 0.4177134 & 0.2434896 & 0.3704390 & 0.4651104 & 0.8218953 & 0.5782429 & 0.8224783 & 0.5782429 & 0.4651104 & 0.1044283 & 0.6435028 & 0.1214293 & 0.0180450 & 0.0328617 & 0.0178766 & 0.0178899 & 0.0200359 & 0.0181608 & 0.0047686 & 0.0329312 & 0.0047591 & 0.0329312 & 0.0181608 & 0.0044692 & 0.0113081\\
\bottomrule
\end{tabular}
\end{table}

\begin{verbatim}
## Ranger result
## 
## Call:
##  ranger::ranger(dependent.variable.name = ".outcome", data = x,      mtry = param$mtry, min.node.size = param$min.node.size, splitrule = as.character(param$splitrule),      write.forest = TRUE, probability = classProbs, ...) 
## 
## Type:                             Probability estimation 
## Number of trees:                  500 
## Sample size:                      59304 
## Number of independent variables:  4 
## Mtry:                             4 
## Target node size:                 1 
## Variable importance mode:         none 
## Splitrule:                        extratrees 
## OOB prediction error (Brier s.):  4.922144e-05
\end{verbatim}

\begin{verbatim}
## Random Forest 
## 
## 59304 samples
##     4 predictor
##     4 classes: 'PATH_IDLE', 'PATH_MOVING', 'PATH_TRANSITION', 'SHUTTLE' 
## 
## Pre-processing: nearest neighbor imputation (4), centered (4), scaled (4) 
## Resampling: Bootstrapped (10 reps) 
## Summary of sample sizes: 989, 989, 989, 989, 989, 988, ... 
## Resampling results across tuning parameters:
## 
##   mtry  splitrule   logLoss    AUC        prAUC      Accuracy   Kappa    
##   2     gini         2.244775  0.8106723  0.5506284  0.3497281  0.1713700
##   2     extratrees   1.575715  0.8355501  0.5794596  0.3725607  0.2007435
##   3     gini         3.059663  0.7818548  0.5132026  0.3467152  0.1661478
##   3     extratrees   1.382679  0.8680137  0.6345818  0.4104649  0.2375047
##   4     gini        10.645753  0.6912447  0.4210159  0.3439510  0.1625890
##   4     extratrees   1.294548  0.8804000  0.6628977  0.4177134  0.2434896
##   Mean_F1    Mean_Sensitivity  Mean_Specificity  Mean_Pos_Pred_Value
##   0.3057264  0.4023020         0.8017163         0.5101218          
##   0.3163320  0.4176647         0.8112342         0.5745317          
##   0.3120710  0.4037873         0.7999871         0.4884279          
##   0.3550340  0.4521986         0.8208922         0.5865778          
##   0.3123221  0.4096034         0.7985001         0.4730115          
##   0.3704390  0.4651104         0.8218953         0.5782429          
##   Mean_Neg_Pred_Value  Mean_Precision  Mean_Recall  Mean_Detection_Rate
##   0.8025946            0.5101218       0.4023020    0.08743202         
##   0.8123653            0.5745317       0.4176647    0.09314017         
##   0.7995287            0.4884279       0.4037873    0.08667880         
##   0.8213790            0.5865778       0.4521986    0.10261621         
##   0.7975549            0.4730115       0.4096034    0.08598776         
##   0.8224783            0.5782429       0.4651104    0.10442835         
##   Mean_Balanced_Accuracy
##   0.6020092             
##   0.6144494             
##   0.6018872             
##   0.6365454             
##   0.6040517             
##   0.6435028             
## 
## Tuning parameter 'min.node.size' was held constant at a value of 1
## logLoss was used to select the optimal model using the smallest value.
## The final values used for the model were mtry = 4, splitrule =
##  extratrees and min.node.size = 1.
\end{verbatim}

\begin{table}[!h]

\caption{\label{tab:sensor-rf-results}Sensor (IMU 1,3,7,8,9,11) RF - Validation Accuracy}
\centering
\begin{tabular}[t]{lr}
\toprule
  & x\\
\midrule
Accuracy & 1\\
\bottomrule
\end{tabular}
\end{table}

\begin{table}[!h]

\caption{\label{tab:sensor-rf-results}Sensor (IMU 1,3,7,8,9,11) RF Validation Metrics}
\centering
\begin{tabular}[t]{lrrl}
\toprule
  & Sensitivity & Specificity & MultiClassAUC\\
\midrule
PATH\_IDLE & 1 & 1 & 1\\
PATH\_MOVING & 1 & 1 & 1\\
PATH\_TRANSITION & 1 & 1 & 1\\
SHUTTLE & 1 & 1 & 1\\
\bottomrule
\end{tabular}
\end{table}

\begin{table}[!h]

\caption{\label{tab:sensor-rf-results}Sensor (IMU 1,3,7,8,9,11) RF Vaidation Confusion Matrix}
\centering
\begin{tabular}[t]{lrrrr}
\toprule
  & PATH\_IDLE & PATH\_MOVING & PATH\_TRANSITION & SHUTTLE\\
\midrule
PATH\_IDLE & 4524 & 0 & 0 & 0\\
PATH\_MOVING & 0 & 3712 & 0 & 0\\
PATH\_TRANSITION & 0 & 0 & 542 & 0\\
SHUTTLE & 0 & 0 & 0 & 1106\\
\bottomrule
\end{tabular}
\end{table}

\begin{table}[!h]

\caption{\label{tab:sensor-rf-results}Sensor (IMU 1,3,7,8,9,11) RF Vaidation LogLoss}
\centering
\begin{tabular}[t]{r}
\toprule
x\\
\midrule
NA\\
\bottomrule
\end{tabular}
\end{table}

\begin{figure}
\centering
\includegraphics{CHP4_RF_Results_files/figure-latex/plot_sensor-roc-1.pdf}
\caption{Sensor (IMU 1,3,7,8,9,11) ROC using 4 Binary ROC Curves}
\end{figure}


\end{document}
