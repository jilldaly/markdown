\documentclass[]{article}
\usepackage{lmodern}
\usepackage{amssymb,amsmath}
\usepackage{ifxetex,ifluatex}
\usepackage{fixltx2e} % provides \textsubscript
\ifnum 0\ifxetex 1\fi\ifluatex 1\fi=0 % if pdftex
  \usepackage[T1]{fontenc}
  \usepackage[utf8]{inputenc}
\else % if luatex or xelatex
  \ifxetex
    \usepackage{mathspec}
  \else
    \usepackage{fontspec}
  \fi
  \defaultfontfeatures{Ligatures=TeX,Scale=MatchLowercase}
\fi
% use upquote if available, for straight quotes in verbatim environments
\IfFileExists{upquote.sty}{\usepackage{upquote}}{}
% use microtype if available
\IfFileExists{microtype.sty}{%
\usepackage{microtype}
\UseMicrotypeSet[protrusion]{basicmath} % disable protrusion for tt fonts
}{}
\usepackage[margin=1in]{geometry}
\usepackage{hyperref}
\hypersetup{unicode=true,
            pdftitle={Chp 4 Results - Random Forest},
            pdfauthor={Jill Daly},
            pdfborder={0 0 0},
            breaklinks=true}
\urlstyle{same}  % don't use monospace font for urls
\usepackage{graphicx,grffile}
\makeatletter
\def\maxwidth{\ifdim\Gin@nat@width>\linewidth\linewidth\else\Gin@nat@width\fi}
\def\maxheight{\ifdim\Gin@nat@height>\textheight\textheight\else\Gin@nat@height\fi}
\makeatother
% Scale images if necessary, so that they will not overflow the page
% margins by default, and it is still possible to overwrite the defaults
% using explicit options in \includegraphics[width, height, ...]{}
\setkeys{Gin}{width=\maxwidth,height=\maxheight,keepaspectratio}
\IfFileExists{parskip.sty}{%
\usepackage{parskip}
}{% else
\setlength{\parindent}{0pt}
\setlength{\parskip}{6pt plus 2pt minus 1pt}
}
\setlength{\emergencystretch}{3em}  % prevent overfull lines
\providecommand{\tightlist}{%
  \setlength{\itemsep}{0pt}\setlength{\parskip}{0pt}}
\setcounter{secnumdepth}{0}
% Redefines (sub)paragraphs to behave more like sections
\ifx\paragraph\undefined\else
\let\oldparagraph\paragraph
\renewcommand{\paragraph}[1]{\oldparagraph{#1}\mbox{}}
\fi
\ifx\subparagraph\undefined\else
\let\oldsubparagraph\subparagraph
\renewcommand{\subparagraph}[1]{\oldsubparagraph{#1}\mbox{}}
\fi

%%% Use protect on footnotes to avoid problems with footnotes in titles
\let\rmarkdownfootnote\footnote%
\def\footnote{\protect\rmarkdownfootnote}

%%% Change title format to be more compact
\usepackage{titling}

% Create subtitle command for use in maketitle
\newcommand{\subtitle}[1]{
  \posttitle{
    \begin{center}\large#1\end{center}
    }
}

\setlength{\droptitle}{-2em}

  \title{Chp 4 Results - Random Forest}
    \pretitle{\vspace{\droptitle}\centering\huge}
  \posttitle{\par}
    \author{Jill Daly}
    \preauthor{\centering\large\emph}
  \postauthor{\par}
      \predate{\centering\large\emph}
  \postdate{\par}
    \date{05 December, 2018}

\usepackage{booktabs}
\usepackage{longtable}
\usepackage{array}
\usepackage{multirow}
\usepackage[table]{xcolor}
\usepackage{wrapfig}
\usepackage{float}
\usepackage{colortbl}
\usepackage{pdflscape}
\usepackage{tabu}
\usepackage{threeparttable}
\usepackage{threeparttablex}
\usepackage[normalem]{ulem}
\usepackage{makecell}

\begin{document}
\maketitle

\begin{figure}
\centering
\includegraphics{CHP4_RF_Results_files/figure-latex/baseline-rf-plot-1.pdf}
\caption{Baseline Model}
\end{figure}

\begin{table}[!h]

\caption{\label{tab:baseline-rf-params}Baseline RF Model Model Method}
\centering
\begin{tabular}[t]{l}
\toprule
x\\
\midrule
ranger\\
\bottomrule
\end{tabular}
\end{table}

\begin{table}[!h]

\caption{\label{tab:baseline-rf-params}Baseline RF Model Tuning Params}
\centering
\begin{tabular}[t]{lll}
\toprule
parameter & class & label\\
\midrule
mtry & numeric & \#Randomly Selected Predictors\\
splitrule & character & Splitting Rule\\
min.node.size & numeric & Minimal Node Size\\
\bottomrule
\end{tabular}
\end{table}

\begin{table}[!h]

\caption{\label{tab:baseline-rf-params}Baseline RF Best Tuned Model from Caret}
\centering
\begin{tabular}[t]{lrlr}
\toprule
  & mtry & splitrule & min.node.size\\
\midrule
4 & 3 & extratrees & 1\\
\bottomrule
\end{tabular}
\end{table}

\begin{table}[!h]

\caption{\label{tab:baseline-rf-params}Baseline RF Model Coefficients}
\centering
\begin{tabular}[t]{l}
\toprule
x\\
\midrule
Timestamp\\
X\_AXIS\\
Y\_AXIS\\
Z\_AXIS\\
\bottomrule
\end{tabular}
\end{table}

\begin{table}[!h]

\caption{\label{tab:baseline-rf-params}Baseline RF Training Model Results}
\centering
\begin{tabular}[t]{rrlrrrrrrrrrrrrrrrrrrrrrrrrrrrr}
\toprule
mtry & min.node.size & splitrule & logLoss & AUC & prAUC & Accuracy & Kappa & Mean\_F1 & Mean\_Sensitivity & Mean\_Specificity & Mean\_Pos\_Pred\_Value & Mean\_Neg\_Pred\_Value & Mean\_Precision & Mean\_Recall & Mean\_Detection\_Rate & Mean\_Balanced\_Accuracy & logLossSD & AUCSD & prAUCSD & AccuracySD & KappaSD & Mean\_F1SD & Mean\_SensitivitySD & Mean\_SpecificitySD & Mean\_Pos\_Pred\_ValueSD & Mean\_Neg\_Pred\_ValueSD & Mean\_PrecisionSD & Mean\_RecallSD & Mean\_Detection\_RateSD & Mean\_Balanced\_AccuracySD\\
\midrule
2 & 1 & gini & 0.4364861 & 0.9401931 & 0.7348764 & 0.8564669 & 0.7653911 & 0.6798586 & 0.6581450 & 0.9425382 & 0.7535202 & 0.9519964 & 0.7535202 & 0.6581450 & 0.2141167 & 0.8003416 & 0.0190495 & 0.0053625 & 0.0129824 & 0.0064286 & 0.0107000 & 0.0110269 & 0.0099708 & 0.0028525 & 0.0134504 & 0.0029012 & 0.0134504 & 0.0099708 & 0.0016072 & 0.0058544\\
2 & 1 & extratrees & 0.3542531 & 0.9602793 & 0.7959112 & 0.8909828 & 0.8208514 & 0.6995734 & 0.6802416 & 0.9551813 & 0.8146758 & 0.9670750 & 0.8146758 & 0.6802416 & 0.2227457 & 0.8177114 & 0.0101150 & 0.0051265 & 0.0104749 & 0.0050517 & 0.0085072 & 0.0123205 & 0.0104613 & 0.0020249 & 0.0206002 & 0.0017207 & 0.0206002 & 0.0104613 & 0.0012629 & 0.0061375\\
3 & 1 & gini & 0.4963031 & 0.9325785 & 0.7043919 & 0.8458622 & 0.7495179 & 0.6838587 & 0.6613775 & 0.9394433 & 0.7371900 & 0.9473045 & 0.7371900 & 0.6613775 & 0.2114656 & 0.8004104 & 0.0271696 & 0.0053989 & 0.0160997 & 0.0072076 & 0.0115202 & 0.0071003 & 0.0068391 & 0.0030006 & 0.0081959 & 0.0033170 & 0.0081959 & 0.0068391 & 0.0018019 & 0.0041186\\
3 & 1 & extratrees & 0.3452809 & 0.9609352 & 0.7959271 & 0.8901275 & 0.8198121 & 0.7116746 & 0.6877744 & 0.9550502 & 0.8167375 & 0.9662938 & 0.8167375 & 0.6877744 & 0.2225319 & 0.8214123 & 0.0122122 & 0.0057225 & 0.0121544 & 0.0087755 & 0.0146847 & 0.0156189 & 0.0150414 & 0.0035525 & 0.0230981 & 0.0030170 & 0.0230981 & 0.0150414 & 0.0021939 & 0.0091923\\
4 & 1 & gini & 0.6944579 & 0.9192432 & 0.6557814 & 0.8372164 & 0.7366635 & 0.6823386 & 0.6616439 & 0.9370716 & 0.7232106 & 0.9437469 & 0.7232106 & 0.6616439 & 0.2093041 & 0.7993577 & 0.0687442 & 0.0067518 & 0.0290521 & 0.0079375 & 0.0122036 & 0.0092615 & 0.0080195 & 0.0028057 & 0.0159959 & 0.0033580 & 0.0159959 & 0.0080195 & 0.0019844 & 0.0047177\\
4 & 1 & extratrees & 0.3480865 & 0.9600478 & 0.7934735 & 0.8873807 & 0.8155202 & 0.7138522 & 0.6888536 & 0.9541408 & 0.8089383 & 0.9650090 & 0.8089383 & 0.6888536 & 0.2218452 & 0.8214972 & 0.0166105 & 0.0071669 & 0.0168183 & 0.0104253 & 0.0174318 & 0.0189705 & 0.0178781 & 0.0042144 & 0.0263212 & 0.0036187 & 0.0263212 & 0.0178781 & 0.0026063 & 0.0108632\\
\bottomrule
\end{tabular}
\end{table}

\begin{verbatim}
## Ranger result
## 
## Call:
##  ranger::ranger(dependent.variable.name = ".outcome", data = x,      mtry = param$mtry, min.node.size = param$min.node.size, splitrule = as.character(param$splitrule),      write.forest = TRUE, probability = classProbs, ...) 
## 
## Type:                             Probability estimation 
## Number of trees:                  500 
## Sample size:                      4935 
## Number of independent variables:  4 
## Mtry:                             3 
## Target node size:                 1 
## Variable importance mode:         none 
## Splitrule:                        extratrees 
## OOB prediction error (Brier s.):  0.02660776
\end{verbatim}

\begin{verbatim}
## Random Forest 
## 
## 4935 samples
##    4 predictor
##    4 classes: 'PATH_IDLE', 'PATH_MOVING', 'PATH_TRANSITION', 'SHUTTLE' 
## 
## Pre-processing: centered (4), scaled (4) 
## Resampling: Bootstrapped (10 reps) 
## Summary of sample sizes: 493, 493, 494, 492, 493, 494, ... 
## Resampling results across tuning parameters:
## 
##   mtry  splitrule   logLoss    AUC        prAUC      Accuracy   Kappa    
##   2     gini        0.4364861  0.9401931  0.7348764  0.8564669  0.7653911
##   2     extratrees  0.3542531  0.9602793  0.7959112  0.8909828  0.8208514
##   3     gini        0.4963031  0.9325785  0.7043919  0.8458622  0.7495179
##   3     extratrees  0.3452809  0.9609352  0.7959271  0.8901275  0.8198121
##   4     gini        0.6944579  0.9192432  0.6557814  0.8372164  0.7366635
##   4     extratrees  0.3480865  0.9600478  0.7934735  0.8873807  0.8155202
##   Mean_F1    Mean_Sensitivity  Mean_Specificity  Mean_Pos_Pred_Value
##   0.6798586  0.6581450         0.9425382         0.7535202          
##   0.6995734  0.6802416         0.9551813         0.8146758          
##   0.6838587  0.6613775         0.9394433         0.7371900          
##   0.7116746  0.6877744         0.9550502         0.8167375          
##   0.6823386  0.6616439         0.9370716         0.7232106          
##   0.7138522  0.6888536         0.9541408         0.8089383          
##   Mean_Neg_Pred_Value  Mean_Precision  Mean_Recall  Mean_Detection_Rate
##   0.9519964            0.7535202       0.6581450    0.2141167          
##   0.9670750            0.8146758       0.6802416    0.2227457          
##   0.9473045            0.7371900       0.6613775    0.2114656          
##   0.9662938            0.8167375       0.6877744    0.2225319          
##   0.9437469            0.7232106       0.6616439    0.2093041          
##   0.9650090            0.8089383       0.6888536    0.2218452          
##   Mean_Balanced_Accuracy
##   0.8003416             
##   0.8177114             
##   0.8004104             
##   0.8214123             
##   0.7993577             
##   0.8214972             
## 
## Tuning parameter 'min.node.size' was held constant at a value of 1
## logLoss was used to select the optimal model using the smallest value.
## The final values used for the model were mtry = 3, splitrule =
##  extratrees and min.node.size = 1.
\end{verbatim}

\begin{table}[!h]

\caption{\label{tab:baseline-rf-results}Baseline RF - Validation Accuracy}
\centering
\begin{tabular}[t]{lr}
\toprule
  & x\\
\midrule
Accuracy & 0.5797043\\
\bottomrule
\end{tabular}
\end{table}

\begin{table}[!h]

\caption{\label{tab:baseline-rf-results}Baseline RF Validation Metrics}
\centering
\begin{tabular}[t]{lrrl}
\toprule
  & Sensitivity & Specificity & MultiClassAUC\\
\midrule
PATH\_IDLE & 0.1309155 & 1.0000000 & 0.998274339237701\\
PATH\_MOVING & 0.9811015 & 0.6064830 & 0.98066027003091\\
PATH\_TRANSITION & 0.7232472 & 0.9924989 & 0.988391725966124\\
SHUTTLE & 1.0000000 & 0.8115876 & 0.99950192876282\\
\bottomrule
\end{tabular}
\end{table}

\begin{table}[!h]

\caption{\label{tab:baseline-rf-results}Baseline RF Vaidation Confusion Matrix}
\centering
\begin{tabular}[t]{lrrrr}
\toprule
  & PATH\_IDLE & PATH\_MOVING & PATH\_TRANSITION & SHUTTLE\\
\midrule
PATH\_IDLE & 296 & 0 & 0 & 0\\
PATH\_MOVING & 1139 & 1817 & 75 & 0\\
PATH\_TRANSITION & 0 & 35 & 196 & 0\\
SHUTTLE & 826 & 0 & 0 & 553\\
\bottomrule
\end{tabular}
\end{table}

\begin{table}[!h]

\caption{\label{tab:baseline-rf-results}Baseline RF Vaidation LogLoss}
\centering
\begin{tabular}[t]{r}
\toprule
x\\
\midrule
NA\\
\bottomrule
\end{tabular}
\end{table}


\end{document}
