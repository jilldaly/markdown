\documentclass[]{article}
\usepackage{lmodern}
\usepackage{amssymb,amsmath}
\usepackage{ifxetex,ifluatex}
\usepackage{fixltx2e} % provides \textsubscript
\ifnum 0\ifxetex 1\fi\ifluatex 1\fi=0 % if pdftex
  \usepackage[T1]{fontenc}
  \usepackage[utf8]{inputenc}
\else % if luatex or xelatex
  \ifxetex
    \usepackage{mathspec}
  \else
    \usepackage{fontspec}
  \fi
  \defaultfontfeatures{Ligatures=TeX,Scale=MatchLowercase}
\fi
% use upquote if available, for straight quotes in verbatim environments
\IfFileExists{upquote.sty}{\usepackage{upquote}}{}
% use microtype if available
\IfFileExists{microtype.sty}{%
\usepackage{microtype}
\UseMicrotypeSet[protrusion]{basicmath} % disable protrusion for tt fonts
}{}
\usepackage[margin=1in]{geometry}
\usepackage{hyperref}
\hypersetup{unicode=true,
            pdftitle={Chp 4 Results - Random Forest},
            pdfauthor={Jill Daly},
            pdfborder={0 0 0},
            breaklinks=true}
\urlstyle{same}  % don't use monospace font for urls
\usepackage{graphicx,grffile}
\makeatletter
\def\maxwidth{\ifdim\Gin@nat@width>\linewidth\linewidth\else\Gin@nat@width\fi}
\def\maxheight{\ifdim\Gin@nat@height>\textheight\textheight\else\Gin@nat@height\fi}
\makeatother
% Scale images if necessary, so that they will not overflow the page
% margins by default, and it is still possible to overwrite the defaults
% using explicit options in \includegraphics[width, height, ...]{}
\setkeys{Gin}{width=\maxwidth,height=\maxheight,keepaspectratio}
\IfFileExists{parskip.sty}{%
\usepackage{parskip}
}{% else
\setlength{\parindent}{0pt}
\setlength{\parskip}{6pt plus 2pt minus 1pt}
}
\setlength{\emergencystretch}{3em}  % prevent overfull lines
\providecommand{\tightlist}{%
  \setlength{\itemsep}{0pt}\setlength{\parskip}{0pt}}
\setcounter{secnumdepth}{0}
% Redefines (sub)paragraphs to behave more like sections
\ifx\paragraph\undefined\else
\let\oldparagraph\paragraph
\renewcommand{\paragraph}[1]{\oldparagraph{#1}\mbox{}}
\fi
\ifx\subparagraph\undefined\else
\let\oldsubparagraph\subparagraph
\renewcommand{\subparagraph}[1]{\oldsubparagraph{#1}\mbox{}}
\fi

%%% Use protect on footnotes to avoid problems with footnotes in titles
\let\rmarkdownfootnote\footnote%
\def\footnote{\protect\rmarkdownfootnote}

%%% Change title format to be more compact
\usepackage{titling}

% Create subtitle command for use in maketitle
\newcommand{\subtitle}[1]{
  \posttitle{
    \begin{center}\large#1\end{center}
    }
}

\setlength{\droptitle}{-2em}

  \title{Chp 4 Results - Random Forest}
    \pretitle{\vspace{\droptitle}\centering\huge}
  \posttitle{\par}
    \author{Jill Daly}
    \preauthor{\centering\large\emph}
  \postauthor{\par}
      \predate{\centering\large\emph}
  \postdate{\par}
    \date{05 December, 2018}

\usepackage{booktabs}
\usepackage{longtable}
\usepackage{array}
\usepackage{multirow}
\usepackage[table]{xcolor}
\usepackage{wrapfig}
\usepackage{float}
\usepackage{colortbl}
\usepackage{pdflscape}
\usepackage{tabu}
\usepackage{threeparttable}
\usepackage{threeparttablex}
\usepackage[normalem]{ulem}
\usepackage{makecell}

\begin{document}
\maketitle

\begin{figure}
\centering
\includegraphics{CHP4_RF_Results_files/figure-latex/baseline-rf-plot-1.pdf}
\caption{Baseline Model}
\end{figure}

\begin{table}[!h]

\caption{\label{tab:baseline-rf-params}Baseline RF Model Model Method}
\centering
\begin{tabular}[t]{l}
\toprule
x\\
\midrule
ranger\\
\bottomrule
\end{tabular}
\end{table}

\begin{table}[!h]

\caption{\label{tab:baseline-rf-params}Baseline RF Model Tuning Params}
\centering
\begin{tabular}[t]{lll}
\toprule
parameter & class & label\\
\midrule
mtry & numeric & \#Randomly Selected Predictors\\
splitrule & character & Splitting Rule\\
min.node.size & numeric & Minimal Node Size\\
\bottomrule
\end{tabular}
\end{table}

\begin{table}[!h]

\caption{\label{tab:baseline-rf-params}Baseline RF Best Tuned Model from Caret}
\centering
\begin{tabular}[t]{lrlr}
\toprule
  & mtry & splitrule & min.node.size\\
\midrule
4 & 3 & extratrees & 1\\
\bottomrule
\end{tabular}
\end{table}

\begin{table}[!h]

\caption{\label{tab:baseline-rf-params}Baseline RF Model Coefficients}
\centering
\begin{tabular}[t]{l}
\toprule
x\\
\midrule
Timestamp\\
X\_AXIS\\
Y\_AXIS\\
Z\_AXIS\\
\bottomrule
\end{tabular}
\end{table}

\begin{table}[!h]

\caption{\label{tab:baseline-rf-params}Baseline RF Training Model Results}
\centering
\begin{tabular}[t]{rrlrrrrrrrrrrrrrrrrrrrrrrrrrrrr}
\toprule
mtry & min.node.size & splitrule & logLoss & AUC & prAUC & Accuracy & Kappa & Mean\_F1 & Mean\_Sensitivity & Mean\_Specificity & Mean\_Pos\_Pred\_Value & Mean\_Neg\_Pred\_Value & Mean\_Precision & Mean\_Recall & Mean\_Detection\_Rate & Mean\_Balanced\_Accuracy & logLossSD & AUCSD & prAUCSD & AccuracySD & KappaSD & Mean\_F1SD & Mean\_SensitivitySD & Mean\_SpecificitySD & Mean\_Pos\_Pred\_ValueSD & Mean\_Neg\_Pred\_ValueSD & Mean\_PrecisionSD & Mean\_RecallSD & Mean\_Detection\_RateSD & Mean\_Balanced\_AccuracySD\\
\midrule
2 & 1 & gini & 0.4487455 & 0.9387021 & 0.7353786 & 0.8550068 & 0.7634394 & 0.6844051 & 0.6620867 & 0.9420844 & 0.7554920 & 0.9512844 & 0.7554920 & 0.6620867 & 0.2137517 & 0.8020856 & 0.0280067 & 0.0031981 & 0.0132194 & 0.0069373 & 0.0108489 & 0.0143302 & 0.0118781 & 0.0024055 & 0.0294446 & 0.0030692 & 0.0294446 & 0.0118781 & 0.0017343 & 0.0066330\\
2 & 1 & extratrees & 0.3530964 & 0.9617985 & 0.7972860 & 0.8909119 & 0.8208446 & 0.7028694 & 0.6828205 & 0.9551559 & 0.8190249 & 0.9669909 & 0.8190249 & 0.6828205 & 0.2227280 & 0.8189882 & 0.0124065 & 0.0027975 & 0.0086557 & 0.0088995 & 0.0151193 & 0.0171659 & 0.0153807 & 0.0039673 & 0.0286393 & 0.0031075 & 0.0286393 & 0.0153807 & 0.0022249 & 0.0095433\\
3 & 1 & gini & 0.5059330 & 0.9321490 & 0.7072317 & 0.8452941 & 0.7488978 & 0.6861009 & 0.6638492 & 0.9392804 & 0.7425958 & 0.9471307 & 0.7425958 & 0.6638492 & 0.2113235 & 0.8015648 & 0.0500526 & 0.0022093 & 0.0141827 & 0.0078235 & 0.0119076 & 0.0172640 & 0.0145806 & 0.0025524 & 0.0376562 & 0.0034468 & 0.0376562 & 0.0145806 & 0.0019559 & 0.0078804\\
3 & 1 & extratrees & 0.3467689 & 0.9626015 & 0.7973887 & 0.8892257 & 0.8183917 & 0.7104901 & 0.6874635 & 0.9546831 & 0.8155006 & 0.9659645 & 0.8155006 & 0.6874635 & 0.2223064 & 0.8210733 & 0.0111946 & 0.0031310 & 0.0085157 & 0.0090085 & 0.0152882 & 0.0180415 & 0.0162764 & 0.0040344 & 0.0252857 & 0.0031902 & 0.0252857 & 0.0162764 & 0.0022521 & 0.0099721\\
4 & 1 & gini & 0.6604023 & 0.9218684 & 0.6714717 & 0.8355367 & 0.7341041 & 0.6817465 & 0.6613384 & 0.9362863 & 0.7252357 & 0.9430396 & 0.7252357 & 0.6613384 & 0.2088842 & 0.7988124 & 0.0952746 & 0.0029518 & 0.0217877 & 0.0079621 & 0.0119472 & 0.0161945 & 0.0137684 & 0.0024694 & 0.0306039 & 0.0034747 & 0.0306039 & 0.0137684 & 0.0019905 & 0.0075639\\
4 & 1 & extratrees & 0.3498294 & 0.9608276 & 0.7905987 & 0.8848190 & 0.8113087 & 0.7106646 & 0.6858637 & 0.9530537 & 0.8096822 & 0.9640836 & 0.8096822 & 0.6858637 & 0.2212047 & 0.8194587 & 0.0118728 & 0.0035990 & 0.0105722 & 0.0092255 & 0.0155113 & 0.0164312 & 0.0161502 & 0.0039776 & 0.0262108 & 0.0032694 & 0.0262108 & 0.0161502 & 0.0023064 & 0.0097932\\
\bottomrule
\end{tabular}
\end{table}

\begin{verbatim}
## Ranger result
## 
## Call:
##  ranger::ranger(dependent.variable.name = ".outcome", data = x,      mtry = param$mtry, min.node.size = param$min.node.size, splitrule = as.character(param$splitrule),      write.forest = TRUE, probability = classProbs, ...) 
## 
## Type:                             Probability estimation 
## Number of trees:                  500 
## Sample size:                      4942 
## Number of independent variables:  4 
## Mtry:                             3 
## Target node size:                 1 
## Variable importance mode:         none 
## Splitrule:                        extratrees 
## OOB prediction error (Brier s.):  0.0243982
\end{verbatim}

\begin{table}[!h]

\caption{\label{tab:baseline-rf-results}Baseline RF - Validation Accuracy}
\centering
\begin{tabular}[t]{lr}
\toprule
  & x\\
\midrule
Accuracy & 0.6157426\\
\bottomrule
\end{tabular}
\end{table}

\begin{table}[!h]

\caption{\label{tab:baseline-rf-results}Baseline RF Validation Metrics}
\centering
\begin{tabular}[t]{lrrl}
\toprule
  & Sensitivity & Specificity & MultiClassAUC\\
\midrule
PATH\_IDLE & 0.2046861 & 1.0000000 & 0.998426798368898\\
PATH\_MOVING & 0.9811422 & 0.6386909 & 0.984004863454533\\
PATH\_TRANSITION & 0.7601476 & 0.9925070 & 0.991583856108311\\
SHUTTLE & 1.0000000 & 0.8293461 & 0.999448110659684\\
\bottomrule
\end{tabular}
\end{table}

\begin{table}[!h]

\caption{\label{tab:baseline-rf-results}Baseline RF Vaidation Confusion Matrix}
\centering
\begin{tabular}[t]{lrrrr}
\toprule
  & PATH\_IDLE & PATH\_MOVING & PATH\_TRANSITION & SHUTTLE\\
\midrule
PATH\_IDLE & 463 & 0 & 0 & 0\\
PATH\_MOVING & 1050 & 1821 & 65 & 0\\
PATH\_TRANSITION & 0 & 35 & 206 & 0\\
SHUTTLE & 749 & 0 & 0 & 553\\
\bottomrule
\end{tabular}
\end{table}

\begin{table}[!h]

\caption{\label{tab:baseline-rf-results}Baseline RF Vaidation LogLoss}
\centering
\begin{tabular}[t]{r}
\toprule
x\\
\midrule
NA\\
\bottomrule
\end{tabular}
\end{table}


\end{document}
