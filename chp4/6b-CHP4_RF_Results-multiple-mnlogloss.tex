\documentclass[]{article}
\usepackage{lmodern}
\usepackage{amssymb,amsmath}
\usepackage{ifxetex,ifluatex}
\usepackage{fixltx2e} % provides \textsubscript
\ifnum 0\ifxetex 1\fi\ifluatex 1\fi=0 % if pdftex
  \usepackage[T1]{fontenc}
  \usepackage[utf8]{inputenc}
\else % if luatex or xelatex
  \ifxetex
    \usepackage{mathspec}
  \else
    \usepackage{fontspec}
  \fi
  \defaultfontfeatures{Ligatures=TeX,Scale=MatchLowercase}
\fi
% use upquote if available, for straight quotes in verbatim environments
\IfFileExists{upquote.sty}{\usepackage{upquote}}{}
% use microtype if available
\IfFileExists{microtype.sty}{%
\usepackage{microtype}
\UseMicrotypeSet[protrusion]{basicmath} % disable protrusion for tt fonts
}{}
\usepackage[margin=1in]{geometry}
\usepackage{hyperref}
\hypersetup{unicode=true,
            pdftitle={Chp 4 Results - Random Forest},
            pdfauthor={Jill Daly},
            pdfborder={0 0 0},
            breaklinks=true}
\urlstyle{same}  % don't use monospace font for urls
\usepackage{graphicx,grffile}
\makeatletter
\def\maxwidth{\ifdim\Gin@nat@width>\linewidth\linewidth\else\Gin@nat@width\fi}
\def\maxheight{\ifdim\Gin@nat@height>\textheight\textheight\else\Gin@nat@height\fi}
\makeatother
% Scale images if necessary, so that they will not overflow the page
% margins by default, and it is still possible to overwrite the defaults
% using explicit options in \includegraphics[width, height, ...]{}
\setkeys{Gin}{width=\maxwidth,height=\maxheight,keepaspectratio}
\IfFileExists{parskip.sty}{%
\usepackage{parskip}
}{% else
\setlength{\parindent}{0pt}
\setlength{\parskip}{6pt plus 2pt minus 1pt}
}
\setlength{\emergencystretch}{3em}  % prevent overfull lines
\providecommand{\tightlist}{%
  \setlength{\itemsep}{0pt}\setlength{\parskip}{0pt}}
\setcounter{secnumdepth}{0}
% Redefines (sub)paragraphs to behave more like sections
\ifx\paragraph\undefined\else
\let\oldparagraph\paragraph
\renewcommand{\paragraph}[1]{\oldparagraph{#1}\mbox{}}
\fi
\ifx\subparagraph\undefined\else
\let\oldsubparagraph\subparagraph
\renewcommand{\subparagraph}[1]{\oldsubparagraph{#1}\mbox{}}
\fi

%%% Use protect on footnotes to avoid problems with footnotes in titles
\let\rmarkdownfootnote\footnote%
\def\footnote{\protect\rmarkdownfootnote}

%%% Change title format to be more compact
\usepackage{titling}

% Create subtitle command for use in maketitle
\newcommand{\subtitle}[1]{
  \posttitle{
    \begin{center}\large#1\end{center}
    }
}

\setlength{\droptitle}{-2em}

  \title{Chp 4 Results - Random Forest}
    \pretitle{\vspace{\droptitle}\centering\huge}
  \posttitle{\par}
    \author{Jill Daly}
    \preauthor{\centering\large\emph}
  \postauthor{\par}
      \predate{\centering\large\emph}
  \postdate{\par}
    \date{06 December, 2018}

\usepackage{booktabs}
\usepackage{longtable}
\usepackage{array}
\usepackage{multirow}
\usepackage[table]{xcolor}
\usepackage{wrapfig}
\usepackage{float}
\usepackage{colortbl}
\usepackage{pdflscape}
\usepackage{tabu}
\usepackage{threeparttable}
\usepackage{threeparttablex}
\usepackage[normalem]{ulem}
\usepackage{makecell}

\begin{document}
\maketitle

\begin{figure}
\centering
\includegraphics{CHP4_RF_Results_files/figure-latex/baseline-rf-plot-1.pdf}
\caption{Baseline Vibration Model}
\end{figure}

\begin{table}[!h]

\caption{\label{tab:baseline-rf-params}Baseline Vibration RF Model Model Method}
\centering
\begin{tabular}[t]{l}
\toprule
x\\
\midrule
ranger\\
\bottomrule
\end{tabular}
\end{table}

\begin{table}[!h]

\caption{\label{tab:baseline-rf-params}Baseline Vibration RF Model Tuning Params}
\centering
\begin{tabular}[t]{lll}
\toprule
parameter & class & label\\
\midrule
mtry & numeric & \#Randomly Selected Predictors\\
splitrule & character & Splitting Rule\\
min.node.size & numeric & Minimal Node Size\\
\bottomrule
\end{tabular}
\end{table}

\begin{table}[!h]

\caption{\label{tab:baseline-rf-params}Baseline Vibration RF Best Tuned Model from Caret}
\centering
\begin{tabular}[t]{lrlr}
\toprule
  & mtry & splitrule & min.node.size\\
\midrule
6 & 4 & extratrees & 1\\
\bottomrule
\end{tabular}
\end{table}

\begin{table}[!h]

\caption{\label{tab:baseline-rf-params}Baseline Vibration RF Model Coefficients}
\centering
\begin{tabular}[t]{l}
\toprule
x\\
\midrule
Timestamp\\
X\_AXIS\\
Y\_AXIS\\
Z\_AXIS\\
\bottomrule
\end{tabular}
\end{table}

\begin{table}[!h]

\caption{\label{tab:baseline-rf-params}Baseline Vibration RF Training Model Results}
\centering
\begin{tabular}[t]{rrlrr}
\toprule
mtry & min.node.size & splitrule & logLoss & logLossSD\\
\midrule
2 & 1 & gini & 0.4345940 & 0.0256439\\
2 & 1 & extratrees & 0.3516809 & 0.0134467\\
3 & 1 & gini & 0.4880673 & 0.0271524\\
3 & 1 & extratrees & 0.3436974 & 0.0182447\\
4 & 1 & gini & 0.6376795 & 0.0799568\\
4 & 1 & extratrees & 0.3435414 & 0.0201902\\
\bottomrule
\end{tabular}
\end{table}

\begin{verbatim}
## Ranger result
## 
## Call:
##  ranger::ranger(dependent.variable.name = ".outcome", data = x,      mtry = param$mtry, min.node.size = param$min.node.size, splitrule = as.character(param$splitrule),      write.forest = TRUE, probability = classProbs, ...) 
## 
## Type:                             Probability estimation 
## Number of trees:                  500 
## Sample size:                      4942 
## Number of independent variables:  4 
## Mtry:                             4 
## Target node size:                 1 
## Variable importance mode:         none 
## Splitrule:                        extratrees 
## OOB prediction error (Brier s.):  0.0236973
\end{verbatim}

\begin{verbatim}
## Random Forest 
## 
## 4942 samples
##    4 predictor
##    4 classes: 'PATH_IDLE', 'PATH_MOVING', 'PATH_TRANSITION', 'SHUTTLE' 
## 
## Pre-processing: nearest neighbor imputation (4), centered (4), scaled (4) 
## Resampling: Bootstrapped (10 reps) 
## Summary of sample sizes: 494, 494, 495, 495, 494, 494, ... 
## Resampling results across tuning parameters:
## 
##   mtry  splitrule   logLoss  
##   2     gini        0.4345940
##   2     extratrees  0.3516809
##   3     gini        0.4880673
##   3     extratrees  0.3436974
##   4     gini        0.6376795
##   4     extratrees  0.3435414
## 
## Tuning parameter 'min.node.size' was held constant at a value of 1
## logLoss was used to select the optimal model using the smallest value.
## The final values used for the model were mtry = 4, splitrule =
##  extratrees and min.node.size = 1.
\end{verbatim}

\begin{table}[!h]

\caption{\label{tab:baseline-rf-results}Baseline Vibration RF - Validation Accuracy}
\centering
\begin{tabular}[t]{lr}
\toprule
  & x\\
\midrule
Accuracy & 0.5918656\\
\bottomrule
\end{tabular}
\end{table}

\begin{table}[!h]

\caption{\label{tab:baseline-rf-results}Baseline Vibration RF Validation Metrics}
\centering
\begin{tabular}[t]{lrrl}
\toprule
  & Sensitivity & Specificity & MultiClassAUC\\
\midrule
PATH\_IDLE & 0.1414677 & 1.0000000 & 0.998044343930216\\
PATH\_MOVING & 0.9838362 & 0.6092029 & 0.988328739217154\\
PATH\_TRANSITION & 0.8339483 & 0.9935774 & 0.99421372826445\\
SHUTTLE & 1.0000000 & 0.8220551 & 0.999112733337536\\
\bottomrule
\end{tabular}
\end{table}

\begin{table}[!h]

\caption{\label{tab:baseline-rf-results}Baseline Vibration RF Vaidation Confusion Matrix}
\centering
\begin{tabular}[t]{lrrrr}
\toprule
  & PATH\_IDLE & PATH\_MOVING & PATH\_TRANSITION & SHUTTLE\\
\midrule
PATH\_IDLE & 320 & 0 & 0 & 0\\
PATH\_MOVING & 1161 & 1826 & 45 & 0\\
PATH\_TRANSITION & 0 & 30 & 226 & 0\\
SHUTTLE & 781 & 0 & 0 & 553\\
\bottomrule
\end{tabular}
\end{table}

\begin{table}[!h]

\caption{\label{tab:baseline-rf-results}Baseline Vibration RF Vaidation LogLoss}
\centering
\begin{tabular}[t]{r}
\toprule
x\\
\midrule
NA\\
\bottomrule
\end{tabular}
\end{table}

\begin{figure}
\centering
\includegraphics{CHP4_RF_Results_files/figure-latex/plot_mc-roc-1.pdf}
\caption{Baseline Vibration ROC using 4 Binary ROC Curves}
\end{figure}

\begin{figure}
\centering
\includegraphics{CHP4_RF_Results_files/figure-latex/mag-rf-plot-1.pdf}
\caption{Mag Model}
\end{figure}

\begin{table}[!h]

\caption{\label{tab:mag-rf-params}Mag RF Training Model Results}
\centering
\begin{tabular}[t]{rrlrr}
\toprule
mtry & min.node.size & splitrule & logLoss & logLossSD\\
\midrule
2 & 1 & gini & 0.2092190 & 0.0171072\\
2 & 1 & extratrees & 0.2111362 & 0.0055793\\
3 & 1 & gini & 0.2324539 & 0.0309970\\
3 & 1 & extratrees & 0.2064035 & 0.0071584\\
4 & 1 & gini & 0.2525624 & 0.0355965\\
4 & 1 & extratrees & 0.2032366 & 0.0077632\\
\bottomrule
\end{tabular}
\end{table}

\begin{verbatim}
## Ranger result
## 
## Call:
##  ranger::ranger(dependent.variable.name = ".outcome", data = x,      mtry = param$mtry, min.node.size = param$min.node.size, splitrule = as.character(param$splitrule),      write.forest = TRUE, probability = classProbs, ...) 
## 
## Type:                             Probability estimation 
## Number of trees:                  500 
## Sample size:                      4942 
## Number of independent variables:  4 
## Mtry:                             4 
## Target node size:                 1 
## Variable importance mode:         none 
## Splitrule:                        extratrees 
## OOB prediction error (Brier s.):  0.01622868
\end{verbatim}

\begin{verbatim}
## Random Forest 
## 
## 4942 samples
##    4 predictor
##    4 classes: 'PATH_IDLE', 'PATH_MOVING', 'PATH_TRANSITION', 'SHUTTLE' 
## 
## Pre-processing: nearest neighbor imputation (4), centered (4), scaled (4) 
## Resampling: Bootstrapped (10 reps) 
## Summary of sample sizes: 494, 494, 495, 495, 494, 494, ... 
## Resampling results across tuning parameters:
## 
##   mtry  splitrule   logLoss  
##   2     gini        0.2092190
##   2     extratrees  0.2111362
##   3     gini        0.2324539
##   3     extratrees  0.2064035
##   4     gini        0.2525624
##   4     extratrees  0.2032366
## 
## Tuning parameter 'min.node.size' was held constant at a value of 1
## logLoss was used to select the optimal model using the smallest value.
## The final values used for the model were mtry = 4, splitrule =
##  extratrees and min.node.size = 1.
\end{verbatim}

\begin{table}[!h]

\caption{\label{tab:mag-rf-results}Mag RF - Validation Accuracy}
\centering
\begin{tabular}[t]{lr}
\toprule
  & x\\
\midrule
Accuracy & 0.4196682\\
\bottomrule
\end{tabular}
\end{table}

\begin{table}[!h]

\caption{\label{tab:mag-rf-results}Mag RF Validation Metrics}
\centering
\begin{tabular}[t]{lrrl}
\toprule
  & Sensitivity & Specificity & MultiClassAUC\\
\midrule
PATH\_IDLE & 0.0000000 & 1.0000000 & 0.992485599192367\\
PATH\_MOVING & 0.7052802 & 0.5194426 & 0.725379459796187\\
PATH\_TRANSITION & 0.8929889 & 0.9154357 & 0.942341889700207\\
SHUTTLE & 0.9457505 & 0.7744361 & 0.978250121440375\\
\bottomrule
\end{tabular}
\end{table}

\begin{table}[!h]

\caption{\label{tab:mag-rf-results}Mag RF Vaidation Confusion Matrix}
\centering
\begin{tabular}[t]{lrrrr}
\toprule
  & PATH\_IDLE & PATH\_MOVING & PATH\_TRANSITION & SHUTTLE\\
\midrule
PATH\_IDLE & 0 & 0 & 0 & 0\\
PATH\_MOVING & 1438 & 1309 & 20 & 25\\
PATH\_TRANSITION & 4 & 386 & 242 & 5\\
SHUTTLE & 820 & 161 & 9 & 523\\
\bottomrule
\end{tabular}
\end{table}

\begin{table}[!h]

\caption{\label{tab:mag-rf-results}Mag RF Vaidation LogLoss}
\centering
\begin{tabular}[t]{r}
\toprule
x\\
\midrule
NA\\
\bottomrule
\end{tabular}
\end{table}

\begin{figure}
\centering
\includegraphics{CHP4_RF_Results_files/figure-latex/plot_mag-roc-1.pdf}
\caption{Mag ROC using 4 Binary ROC Curves}
\end{figure}

\begin{figure}
\centering
\includegraphics{CHP4_RF_Results_files/figure-latex/combined-rf-plot-1.pdf}
\caption{Vibration + Magnetometer Model}
\end{figure}

\begin{table}[!h]

\caption{\label{tab:combined-rf-params}Vibration + Magnetometer RF Training Model Results}
\centering
\begin{tabular}[t]{rrlrr}
\toprule
mtry & min.node.size & splitrule & logLoss & logLossSD\\
\midrule
2 & 1 & gini & 1.7195396 & 0.5881610\\
2 & 1 & extratrees & 1.0850881 & 0.0517257\\
3 & 1 & gini & 1.9053229 & 0.6463260\\
3 & 1 & extratrees & 1.0029054 & 0.0470129\\
4 & 1 & gini & 4.0143578 & 1.8836691\\
4 & 1 & extratrees & 0.9645406 & 0.0613808\\
\bottomrule
\end{tabular}
\end{table}

\begin{verbatim}
## Ranger result
## 
## Call:
##  ranger::ranger(dependent.variable.name = ".outcome", data = x,      mtry = param$mtry, min.node.size = param$min.node.size, splitrule = as.character(param$splitrule),      write.forest = TRUE, probability = classProbs, ...) 
## 
## Type:                             Probability estimation 
## Number of trees:                  500 
## Sample size:                      9884 
## Number of independent variables:  4 
## Mtry:                             4 
## Target node size:                 1 
## Variable importance mode:         none 
## Splitrule:                        extratrees 
## OOB prediction error (Brier s.):  0.008035631
\end{verbatim}

\begin{verbatim}
## Random Forest 
## 
## 9884 samples
##    4 predictor
##    4 classes: 'PATH_IDLE', 'PATH_MOVING', 'PATH_TRANSITION', 'SHUTTLE' 
## 
## Pre-processing: nearest neighbor imputation (4), centered (4), scaled (4) 
## Resampling: Bootstrapped (10 reps) 
## Summary of sample sizes: 494, 494, 495, 495, 494, 494, ... 
## Resampling results across tuning parameters:
## 
##   mtry  splitrule   logLoss  
##   2     gini        1.7195396
##   2     extratrees  1.0850881
##   3     gini        1.9053229
##   3     extratrees  1.0029054
##   4     gini        4.0143578
##   4     extratrees  0.9645406
## 
## Tuning parameter 'min.node.size' was held constant at a value of 1
## logLoss was used to select the optimal model using the smallest value.
## The final values used for the model were mtry = 4, splitrule =
##  extratrees and min.node.size = 1.
\end{verbatim}

\begin{table}[!h]

\caption{\label{tab:combined-rf-results}Vibration + Magnetometer RF - Validation Accuracy}
\centering
\begin{tabular}[t]{lr}
\toprule
  & x\\
\midrule
Accuracy & 0.8954877\\
\bottomrule
\end{tabular}
\end{table}

\begin{table}[!h]

\caption{\label{tab:combined-rf-results}Vibration + Magnetometer RF Validation Metrics}
\centering
\begin{tabular}[t]{lrrl}
\toprule
  & Sensitivity & Specificity & MultiClassAUC\\
\midrule
PATH\_IDLE & 0.8231653 & 1.0000000 & 0.999839826068596\\
PATH\_MOVING & 0.9399246 & 0.9536617 & 0.99632325211746\\
PATH\_TRANSITION & 0.9815498 & 0.9728110 & 0.997001301901266\\
SHUTTLE & 1.0000000 & 0.9438369 & 0.999961579932076\\
\bottomrule
\end{tabular}
\end{table}

\begin{table}[!h]

\caption{\label{tab:combined-rf-results}Vibration + Magnetometer RF Vaidation Confusion Matrix}
\centering
\begin{tabular}[t]{lrrrr}
\toprule
  & PATH\_IDLE & PATH\_MOVING & PATH\_TRANSITION & SHUTTLE\\
\midrule
PATH\_IDLE & 3724 & 0 & 0 & 0\\
PATH\_MOVING & 276 & 3489 & 10 & 0\\
PATH\_TRANSITION & 31 & 223 & 532 & 0\\
SHUTTLE & 493 & 0 & 0 & 1106\\
\bottomrule
\end{tabular}
\end{table}

\begin{table}[!h]

\caption{\label{tab:combined-rf-results}Vibration + Magnetometer RF Vaidation LogLoss}
\centering
\begin{tabular}[t]{r}
\toprule
x\\
\midrule
NA\\
\bottomrule
\end{tabular}
\end{table}

\begin{figure}
\centering
\includegraphics{CHP4_RF_Results_files/figure-latex/plot_combined-roc-1.pdf}
\caption{Vibration + Magnetometer ROC using 4 Binary ROC Curves}
\end{figure}

\begin{figure}
\centering
\includegraphics{CHP4_RF_Results_files/figure-latex/a_1_v_0_04-rf-plot-1.pdf}
\caption{a\_1\_v\_0\_04 Model}
\end{figure}

\textbackslash{}begin\{table\}{[}!h{]}

\textbackslash{}caption\{\label{tab:a_1_v_0_04-rf-params}a\_1\_v\_0\_04
RF Training Model Results\} \centering

\begin{tabular}[t]{rrlrr}
\toprule
mtry & min.node.size & splitrule & logLoss & logLossSD\\
\midrule
2 & 1 & gini & 0.7796189 & 0.1054723\\
2 & 1 & extratrees & 0.6393502 & 0.0288047\\
3 & 1 & gini & 0.8536428 & 0.1186917\\
3 & 1 & extratrees & 0.6379508 & 0.0282093\\
4 & 1 & gini & 0.9332342 & 0.1879865\\
4 & 1 & extratrees & 0.6441374 & 0.0382260\\
\bottomrule
\end{tabular}

\textbackslash{}end\{table\}

\begin{verbatim}
## Ranger result
## 
## Call:
##  ranger::ranger(dependent.variable.name = ".outcome", data = x,      mtry = param$mtry, min.node.size = param$min.node.size, splitrule = as.character(param$splitrule),      write.forest = TRUE, probability = classProbs, ...) 
## 
## Type:                             Probability estimation 
## Number of trees:                  500 
## Sample size:                      8362 
## Number of independent variables:  4 
## Mtry:                             3 
## Target node size:                 1 
## Variable importance mode:         none 
## Splitrule:                        extratrees 
## OOB prediction error (Brier s.):  0.01326603
\end{verbatim}

\begin{verbatim}
## Random Forest 
## 
## 8362 samples
##    4 predictor
##    4 classes: 'PATH_IDLE', 'PATH_MOVING', 'PATH_TRANSITION', 'SHUTTLE' 
## 
## Pre-processing: nearest neighbor imputation (4), centered (4), scaled (4) 
## Resampling: Bootstrapped (10 reps) 
## Summary of sample sizes: 494, 494, 495, 495, 494, 494, ... 
## Resampling results across tuning parameters:
## 
##   mtry  splitrule   logLoss  
##   2     gini        0.7796189
##   2     extratrees  0.6393502
##   3     gini        0.8536428
##   3     extratrees  0.6379508
##   4     gini        0.9332342
##   4     extratrees  0.6441374
## 
## Tuning parameter 'min.node.size' was held constant at a value of 1
## logLoss was used to select the optimal model using the smallest value.
## The final values used for the model were mtry = 3, splitrule =
##  extratrees and min.node.size = 1.
\end{verbatim}

\textbackslash{}begin\{table\}{[}!h{]}

\textbackslash{}caption\{\label{tab:a_1_v_0_04-rf-params}a\_1\_v\_0\_04
RF - Validation Accuracy\} \centering

\begin{tabular}[t]{lr}
\toprule
  & x\\
\midrule
Accuracy & 0.5875389\\
\bottomrule
\end{tabular}

\textbackslash{}end\{table\}

\textbackslash{}begin\{table\}{[}!h{]}

\textbackslash{}caption\{\label{tab:a_1_v_0_04-rf-params}a\_1\_v\_0\_04
RF Validation Metrics\} \centering

\begin{tabular}[t]{lrrl}
\toprule
  & Sensitivity & Specificity & MultiClassAUC\\
\midrule
PATH\_IDLE & 0.3970301 & 0.9996917 & 0.997810940473145\\
PATH\_MOVING & 0.8532505 & 0.7456140 & 0.936670646283709\\
PATH\_TRANSITION & 0.7800000 & 0.9621548 & 0.975789201183432\\
SHUTTLE & 0.9988914 & 0.7927614 & 0.999069018505198\\
\bottomrule
\end{tabular}

\textbackslash{}end\{table\}

\begin{figure}
\centering
\includegraphics{CHP4_RF_Results_files/figure-latex/a_1_v_0_06-rf-plot-1.pdf}
\caption{a\_1\_v\_0\_06 Model}
\end{figure}

\textbackslash{}begin\{table\}{[}!h{]}

\textbackslash{}caption\{\label{tab:a_1_v_0_06-rf-params}a\_1\_v\_0\_06
RF Training Model Results\} \centering

\begin{tabular}[t]{rrlrr}
\toprule
mtry & min.node.size & splitrule & logLoss & logLossSD\\
\midrule
2 & 1 & gini & 1.404503 & 0.1638134\\
2 & 1 & extratrees & 1.581914 & 0.5605284\\
3 & 1 & gini & 2.359028 & 0.5717812\\
3 & 1 & extratrees & 1.647546 & 0.5409843\\
4 & 1 & gini & 6.054099 & 0.9444707\\
4 & 1 & extratrees & 1.965166 & 0.5644683\\
\bottomrule
\end{tabular}

\textbackslash{}end\{table\}

\begin{verbatim}
## Ranger result
## 
## Call:
##  ranger::ranger(dependent.variable.name = ".outcome", data = x,      mtry = param$mtry, min.node.size = param$min.node.size, splitrule = as.character(param$splitrule),      write.forest = TRUE, probability = classProbs, ...) 
## 
## Type:                             Probability estimation 
## Number of trees:                  500 
## Sample size:                      23132 
## Number of independent variables:  4 
## Mtry:                             2 
## Target node size:                 1 
## Variable importance mode:         none 
## Splitrule:                        gini 
## OOB prediction error (Brier s.):  0.009390752
\end{verbatim}

\begin{verbatim}
## Random Forest 
## 
## 23132 samples
##     4 predictor
##     4 classes: 'PATH_IDLE', 'PATH_MOVING', 'PATH_TRANSITION', 'SHUTTLE' 
## 
## Pre-processing: nearest neighbor imputation (4), centered (4), scaled (4) 
## Resampling: Bootstrapped (10 reps) 
## Summary of sample sizes: 494, 494, 495, 495, 494, 494, ... 
## Resampling results across tuning parameters:
## 
##   mtry  splitrule   logLoss 
##   2     gini        1.404503
##   2     extratrees  1.581914
##   3     gini        2.359028
##   3     extratrees  1.647546
##   4     gini        6.054099
##   4     extratrees  1.965166
## 
## Tuning parameter 'min.node.size' was held constant at a value of 1
## logLoss was used to select the optimal model using the smallest value.
## The final values used for the model were mtry = 2, splitrule = gini
##  and min.node.size = 1.
\end{verbatim}

\textbackslash{}begin\{table\}{[}!h{]}

\textbackslash{}caption\{\label{tab:a_1_v_0_06-rf-params}a\_1\_v\_0\_06
RF - Validation Accuracy\} \centering

\begin{tabular}[t]{lr}
\toprule
  & x\\
\midrule
Accuracy & 0.6490144\\
\bottomrule
\end{tabular}

\textbackslash{}end\{table\}

\textbackslash{}begin\{table\}{[}!h{]}

\textbackslash{}caption\{\label{tab:a_1_v_0_06-rf-params}a\_1\_v\_0\_06
RF Validation Metrics\} \centering

\begin{tabular}[t]{lrrl}
\toprule
  & Sensitivity & Specificity & MultiClassAUC\\
\midrule
PATH\_IDLE & 0.3987805 & 0.9996705 & 0.990792075804571\\
PATH\_MOVING & 0.6923801 & 0.7665455 & 0.857084596808021\\
PATH\_TRANSITION & 0.8141862 & 0.7436402 & 0.903621924829476\\
SHUTTLE & 0.8488372 & 0.9617195 & 0.98469590743274\\
\bottomrule
\end{tabular}

\textbackslash{}end\{table\}

\begin{figure}
\centering
\includegraphics{CHP4_RF_Results_files/figure-latex/a_2_v_0_4-rf-plot-1.pdf}
\caption{a\_2\_v\_0\_4 Model}
\end{figure}

\textbackslash{}begin\{table\}{[}!h{]}

\textbackslash{}caption\{\label{tab:a_2_v_0_4-rf-params}a\_2\_v\_0\_4 RF
Training Model Results\} \centering

\begin{tabular}[t]{rrlrr}
\toprule
mtry & min.node.size & splitrule & logLoss & logLossSD\\
\midrule
2 & 1 & gini & 0.7656648 & 0.0501640\\
2 & 1 & extratrees & 0.6557411 & 0.0143744\\
3 & 1 & gini & 0.8426329 & 0.0763494\\
3 & 1 & extratrees & 0.6447617 & 0.0273652\\
4 & 1 & gini & 0.9534549 & 0.1090660\\
4 & 1 & extratrees & 0.6475105 & 0.0308461\\
\bottomrule
\end{tabular}

\textbackslash{}end\{table\}

\begin{verbatim}
## Ranger result
## 
## Call:
##  ranger::ranger(dependent.variable.name = ".outcome", data = x,      mtry = param$mtry, min.node.size = param$min.node.size, splitrule = as.character(param$splitrule),      write.forest = TRUE, probability = classProbs, ...) 
## 
## Type:                             Probability estimation 
## Number of trees:                  500 
## Sample size:                      9064 
## Number of independent variables:  4 
## Mtry:                             3 
## Target node size:                 1 
## Variable importance mode:         none 
## Splitrule:                        extratrees 
## OOB prediction error (Brier s.):  0.01977015
\end{verbatim}

\begin{verbatim}
## Random Forest 
## 
## 9064 samples
##    4 predictor
##    4 classes: 'PATH_IDLE', 'PATH_MOVING', 'PATH_TRANSITION', 'SHUTTLE' 
## 
## Pre-processing: nearest neighbor imputation (4), centered (4), scaled (4) 
## Resampling: Bootstrapped (10 reps) 
## Summary of sample sizes: 494, 494, 495, 495, 494, 494, ... 
## Resampling results across tuning parameters:
## 
##   mtry  splitrule   logLoss  
##   2     gini        0.7656648
##   2     extratrees  0.6557411
##   3     gini        0.8426329
##   3     extratrees  0.6447617
##   4     gini        0.9534549
##   4     extratrees  0.6475105
## 
## Tuning parameter 'min.node.size' was held constant at a value of 1
## logLoss was used to select the optimal model using the smallest value.
## The final values used for the model were mtry = 3, splitrule =
##  extratrees and min.node.size = 1.
\end{verbatim}

\textbackslash{}begin\{table\}{[}!h{]}

\textbackslash{}caption\{\label{tab:a_2_v_0_4-rf-params}a\_2\_v\_0\_4 RF
- Validation Accuracy\} \centering

\begin{tabular}[t]{lr}
\toprule
  & x\\
\midrule
Accuracy & 0.515887\\
\bottomrule
\end{tabular}

\textbackslash{}end\{table\}

\textbackslash{}begin\{table\}{[}!h{]}

\textbackslash{}caption\{\label{tab:a_2_v_0_4-rf-params}a\_2\_v\_0\_4 RF
Validation Metrics\} \centering

\begin{tabular}[t]{lrrl}
\toprule
  & Sensitivity & Specificity & MultiClassAUC\\
\midrule
PATH\_IDLE & 0.2748220 & 0.9994632 & 0.991272154779222\\
PATH\_MOVING & 0.8101105 & 0.7670641 & 0.928282866972157\\
PATH\_TRANSITION & 0.7604167 & 0.8931176 & 0.9486139993543\\
SHUTTLE & 0.9990775 & 0.7637845 & 0.997380674934569\\
\bottomrule
\end{tabular}

\textbackslash{}end\{table\}

\begin{figure}
\centering
\includegraphics{CHP4_RF_Results_files/figure-latex/a_2_v_0_06-rf-plot-1.pdf}
\caption{a\_2\_v\_0\_06 Model}
\end{figure}

\textbackslash{}begin\{table\}{[}!h{]}

\textbackslash{}caption\{\label{tab:a_2_v_0_06-rf-params}a\_2\_v\_0\_06
RF Training Model Results\} \centering

\begin{tabular}[t]{rrlrr}
\toprule
mtry & min.node.size & splitrule & logLoss & logLossSD\\
\midrule
2 & 1 & gini & 1.786720 & 0.5718651\\
2 & 1 & extratrees & 1.534895 & 0.2006457\\
3 & 1 & gini & 3.371592 & 0.7382088\\
3 & 1 & extratrees & 1.780658 & 0.3953811\\
4 & 1 & gini & 7.530882 & 0.7433695\\
4 & 1 & extratrees & 2.645977 & 1.1235611\\
\bottomrule
\end{tabular}

\textbackslash{}end\{table\}

\begin{verbatim}
## Ranger result
## 
## Call:
##  ranger::ranger(dependent.variable.name = ".outcome", data = x,      mtry = param$mtry, min.node.size = param$min.node.size, splitrule = as.character(param$splitrule),      write.forest = TRUE, probability = classProbs, ...) 
## 
## Type:                             Probability estimation 
## Number of trees:                  500 
## Sample size:                      23212 
## Number of independent variables:  4 
## Mtry:                             2 
## Target node size:                 1 
## Variable importance mode:         none 
## Splitrule:                        extratrees 
## OOB prediction error (Brier s.):  0.0154965
\end{verbatim}

\begin{verbatim}
## Random Forest 
## 
## 23212 samples
##     4 predictor
##     4 classes: 'PATH_IDLE', 'PATH_MOVING', 'PATH_TRANSITION', 'SHUTTLE' 
## 
## Pre-processing: nearest neighbor imputation (4), centered (4), scaled (4) 
## Resampling: Bootstrapped (10 reps) 
## Summary of sample sizes: 494, 494, 495, 495, 494, 494, ... 
## Resampling results across tuning parameters:
## 
##   mtry  splitrule   logLoss 
##   2     gini        1.786720
##   2     extratrees  1.534895
##   3     gini        3.371592
##   3     extratrees  1.780658
##   4     gini        7.530882
##   4     extratrees  2.645977
## 
## Tuning parameter 'min.node.size' was held constant at a value of 1
## logLoss was used to select the optimal model using the smallest value.
## The final values used for the model were mtry = 2, splitrule =
##  extratrees and min.node.size = 1.
\end{verbatim}

\textbackslash{}begin\{table\}{[}!h{]}

\textbackslash{}caption\{\label{tab:a_2_v_0_06-rf-params}a\_2\_v\_0\_06
RF - Validation Accuracy\} \centering

\begin{tabular}[t]{lr}
\toprule
  & x\\
\midrule
Accuracy & 0.6203688\\
\bottomrule
\end{tabular}

\textbackslash{}end\{table\}

\textbackslash{}begin\{table\}{[}!h{]}

\textbackslash{}caption\{\label{tab:a_2_v_0_06-rf-params}a\_2\_v\_0\_06
RF Validation Metrics\} \centering

\begin{tabular}[t]{lrrl}
\toprule
  & Sensitivity & Specificity & MultiClassAUC\\
\midrule
PATH\_IDLE & 0.0820056 & 0.9997026 & 0.973866235845022\\
PATH\_MOVING & 0.8186150 & 0.6261501 & 0.852037564855777\\
PATH\_TRANSITION & 0.7782931 & 0.8790369 & 0.931781472130508\\
SHUTTLE & 0.9828431 & 0.8816172 & 0.991207585302024\\
\bottomrule
\end{tabular}

\textbackslash{}end\{table\}


\end{document}
